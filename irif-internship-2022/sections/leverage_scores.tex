\section{Leverage scores}
In order to estimate the importance of a row of a matrix relative to the others,
we define its leverage score.
\begin{definition}[Leverage score]\label{def:leverage-score}
Given a matrix $\A \in \mathbb R^{n \times d}$, the leverage score of $a_i^T$,
the $i$-th row of $\A$, is defined as
$$
\tau_i := a_i^T (\bs {A^TA})^+ a_i
$$
where $\A^+$ denotes the Moore-Penrose pseudoinverse of $\A$.
\end{definition}

In the algorithm, approximate leverage scores are computed, \textit{i.e.,\ }leverage scores of $\A$ according to some approximation $\B$, denoted \textit{generalized leverage scores}.

\begin{definition}[Generalized leverage scores]
Let $\A \in \mathbb R ^{n \times d}$ and $\B \in \mathbb R^{m \times d}$. The
leverage score of the $i$-th rows of $\A$ according to $\B$ is
$$
\tau_i^{\B}(\A) =
    \begin{cases}
        a_i^T(\bs{B ^T B})^+a_i & \text{if } a_i \perp \ker \B \\
        \infty & \text{otherwise}
    \end{cases} \, .
$$
\end{definition}

The case distinction comes from the fact that if $a_i \not\perp \ker \B$, the
generalized leverage score would be null, while $a_i$ could be the only row
pointing in its direction, and thus should be kept.

For the sake of conciseness, if $\B = \A$, we denote $\tau_i^{\B}(\A)$ by
$\tau_i(\A)$, and as long as the knowledge $\A$ is not important, we simply
denote $\tau_i(\A)$ by $\tau_i$.

If $\A$ and $\B$ are relatively close, \textit{i.e.\ }$\B$ is an
$\lambda$-spectral approximation of $\A$, it holds that
$$
\frac{1}{\lambda} \A^T\A \preccurlyeq \B^T\B \preccurlyeq \A^T\A \, ,
$$
which we write
$$
\A \approx_\lambda \B \, .
$$
% This yields the following proposition
\begin{theoremEnd}{theorem}[Leverage score approximate]
    Let $\A \in \mathbb R ^{n \times d}$ and $\B \in \mathbb R^{m \times d}$. If
    $\A \approx_\lambda \B$, then
    $$
        \tau_i(\A) \leq \tau_i^{\B}(\A) \leq \lambda \cdot \tau_i(\A) \, .
    $$
\end{theoremEnd}
\begin{proofEnd}
    By definition of $\A \approx_\lambda \B$, we have
$$
    \frac{1}{\lambda} \A^T\A \preccurlyeq \B^T\B \preccurlyeq \A^T\A \, ,
$$
    thus
$$
    (\A^T\A)^+ \preccurlyeq (\B^T\B)^+ \preccurlyeq \lambda \cdot (\A^T\A)^+ \, ,
$$
    which yields, using the definition of the generalized leverage scores and
    by the positive semi-definiteness of each of the matrices,
$$
    \tau_i(\A) \leq \tau_i^{\B}(\A) \leq \lambda \cdot \tau_i(\A) \, .
$$

\end{proofEnd}
