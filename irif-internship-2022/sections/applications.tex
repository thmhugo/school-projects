\section{Application}
There are several applications of matrix sparsification, and the one we chosed
to depict is convex optimization, and more precisely the interior points methods
(IPM). A quick summary on IPM -- with the explicit algorithm -- is provided in
\autoref{ap:ipm}.

We denote herin the distance between two vectors $u, v \in \mathbb R^{n}$ by the
following weighed norm with respect to the operator $Q$, which we denote the
$Q$-induced norm,
\begin{equation}\label{eq:def-h-norm}
\| u - v \|_Q = \sqrt{(u-v)^TQ(u-v)}\, ,
\end{equation}
to quantify the convergence rate. For our purpose, we denote by $x$ the actual
\emph{current} value of Newton's iterations, by $y$ the approximated one, and by
$x'$ and $y'$ the result of a Newton's iteration starting from $x$ and $y$
respectively; the start superscript denotes the minimizer of $\Phi$. We want to
bound $\|y' - x^* \|$, so that it describes a path that converges towards to the
actual minimizer.


\subsection{Approximate Hessian}

Considering a $\lambda$-spectral approximation of $S^{-1}B$, and taking into
account that our Hessian is $H = B^TS^{-2}B$, we denote the approximated Hessian
$\tilde H$ for some $\lambda \geq 1$. First, note that this implies
\begin{equation}\label{eq:hessian-bounds}
    \begin{aligned}
        & \frac{1}{\lambda} H \preccurlyeq \tilde H \preccurlyeq H \\
\Leftrightarrow \quad & H^{-1} \preccurlyeq \tilde H^{-1} \preccurlyeq \lambda H^{-1} \, .
    \end{aligned}
\end{equation}
At first, we consider an iteration  where the initial points are equal. In order
to use the above inequality, we need to express $\| x' - y' \|$ in terms of
both Hessians, \emph{i.e.,} considering Newton's step. That is,
\begin{equation} \label{eq:vector-distance}
    \begin{aligned}
        \left\| x' - y' \right\|
            & = \left\| \Big(x - H^{-1}\nabla\Phi_\mu(x)\Big) - \Big(y - \tilde H^{-1}\nabla\Phi_\mu(y)\Big)\right\| \\
            & = \left\| x - H^{-1}\nabla\Phi_\mu(x) - x + \tilde H^{-1}\nabla\Phi_\mu(x)\right\| \\
            & = \left\| \tilde H^{-1}\nabla\Phi_\mu(x) - H^{-1}\nabla\Phi_\mu(x) \right\| \\
            & = \left\| \Big[ \tilde H^{-1} - H^{-1}\Big ] \nabla\Phi_\mu(x) \right\| \, . \\
    \end{aligned}
\end{equation}
Thus, it suffices to bound $\big[ \tilde H^{-1} - H^{-1}\big ] $ to obtain
bounds on $\left\| x' - y' \right\|$. A straightforward consequence of
\autoref{eq:hessian-bounds}, is that
\begin{equation*}
\boldsymbol 0 \preccurlyeq \tilde H^{-1} - H^{-1} \preccurlyeq (\lambda - 1) H^{-1} \, ,
\end{equation*}
where $\boldsymbol 0$ is a full-zero matrix.
Therefore, it holds for all vectors, and especially $\nabla \Phi_\mu(x)$, that
\begin{equation*}
    0 \leq \left\| \Big[ \tilde H^{-1} - H^{-1}\Big ] \nabla\Phi_\mu(x) \right\|
 \leq \left\| (\lambda - 1) H^{-1} \nabla\Phi_\mu(x) \right\| \, ,
\end{equation*}

Note that $\left\| H^{-1} \nabla\Phi_\mu(x) \right\|$ is equal to $\|x' - x \|$
\emph{i.e.,} the length of the \emph{actual} Newton's step. As long as $\lambda
< 2$, $y'$ is closer to $x'$ than was $x$. The above equation can thus be
equivalently rewritten thanks to \autoref{eq:vector-distance} as
\begin{equation} \label{eq:x1-y1-bound}
    0 \leq \left\| x' - y' \right\|
 \leq (\lambda - 1) \left\|x' - x \right\| \, .
\end{equation}

In other words, it is possible to bound the distance bewteen $x'$, the actual
point we would like to obtain, and its approximation $y'$ within a
multiplicative factor as long as we consider $\lambda$-spectral approximation of
the Hessian.

In order to prove convergence of the Newton's method when we consider
$\lambda$-spectral approximation of the Hessian, we bound $\|y' - x^* \|$ in
terms of $\|x - x^*\|$. It holds that
\begin{equation}\label{eq:bound-opt-approx-hessian}
    \begin{aligned}
       \| y'-x^*\|
            & =\left\| y' - x' + x' - x^* \right\| \\
            & \leq \left\| y' - x' \| + \| x' - x^* \right\| \\
            & \leq (\lambda - 1) \left\| x - x' \right\| + \frac 1 2 \left\| x - x^*\right\|^2 \\
            & \leq (\lambda - 1) \left\| x - x^* \right\| + \frac{1}{2} \left\| x - x^* \right\| \\
            & = (\lambda - \frac{1}{2}) \left\| x - x^*\right\| & ,\\
    \end{aligned}
\end{equation}
which bounds the step of an approximated iteration. However, this enforces $ 1
\leq \lambda < \frac 3 2$ (\emph{i.e.,} choose $\lambda = 1 + \varepsilon$ for
an $\varepsilon \ll \frac 1 2$) in order to have $\lambda - \frac{1}{2} < 1$ and
hence a step \emph{towards} the minimizer $y^*$ ($y'$ in a \emph{ball} of radius
$ <\| x - x^*\|$ around $x^*$).

\subsection{Gradient approximation}
The goal here is to obtain an approximation of $\nabla \Phi$, the gradient
involved in the interior points method.

\subsubsection{Initial approach}\label{sec:initial-approach}

To do so, we'll consider a vector of random variables, such that its mean is
equal to our gradient, using result of \citeauthor{Cornelissen_2022} which
exposes a quantum algorithm for estimating the mean of a $n$-dimensional random
variable; they provide the following informal theorem.

\begin{theorem}[\cite{Cornelissen_2022}]\label{th:estimate-mean}
    Given a $n$-dimensional random variable $X$, there exists a quantum
    algorithm that returns with high probability an estimate $\tilde \mu$ of the
    mean $\mu$ of $X$ such that
\begin{equation*}
    \| \tilde \mu - \mu \|_2 \leq \frac{\sqrt{n \emph{tr}(\Sigma_X)}}{T}
\end{equation*}
in $\tilde O(T)$ queries as long as $T > n$.
\end{theorem}
Note that the bound is obtained in 2-norm, but the convergence rate of Newton's
method is in terms of the $H$-norm. As usual, we use $B \in \mathbb R^{m \times
n}, s \in \mathbb R^{m}$ and $H = B^T diag(s)^{-2} B$ where we denote $diag(s)$ by $S$. Let $X = B^T
S^{-1}\ket{e}$, where $e \sim \mathcal{U}[m]$. We denote $h := \sum_e B^T S^{-1}
\ket e = B^TS^{-1} \ket{\bs 1} $, where $\ket {\bs 1}$ is the full-one vector,
thus  $\mu_X := \esp X = \frac 1 m B^TS^{-1}\ket{\bs 1} = \frac 1 m h$. We
denote by $\Sigma_Z$ the covariance matrix of the $n$-dimensional random
variable $Z$. The goal is to obtain an $\tilde h$ such that $\|h - \tilde
h\|_{H^{-1}} \leq \varepsilon$ . In other words, we want $\tilde\mu_X$ such that
\begin{equation*}
    \|\mu_X - \tilde \mu_X \|_{H^{-1}} \leq \frac \varepsilon m
\end{equation*}
which, by using \autoref{prop:change-norm}, means
\begin{equation*}
    \left\|H^{-\frac 1 2} \mu_X - H^{-\frac 1 2}\tilde \mu_X \right\|_{2} \leq \frac \varepsilon m \, ;
\end{equation*}
that is,
\begin{equation} \label{eq:estimate-bound}
    \| \mu_Y -  \tilde \mu_Y \|_{2} \leq \frac \varepsilon m \, ,
\end{equation}
for a certain random variable $Y$.
As such, let $Y = H^{-\frac 1 2}X = H^{-\frac 1 2} B^T S^{-1}\ket{e}$ where $e
\sim \mathcal{U}[m]$, hence $\mu_Y = H^{-\frac 1 2} \mu_X$; we use the result of
\cite{Cornelissen_2022} to get the $\tilde \mu_Y$.


The covariance matrix of $Y$ is
\begin{equation*}
\begin{aligned}
\Sigma_Y &= \esp{YY^T} - \esp{Y}\esp{Y^T} \\ &= H^{-\frac 1 2} \Sigma_X H^{-\frac 1 2} - \esp{Y}\esp{Y^T}\, ,
\end{aligned}
\end{equation*}
where
\begin{equation*}
    \Sigma_X = \frac{1}{m} H - \frac{1}{m^2}h h^T
    \quad \text{and} \quad
    \esp{Y} = \frac{1}{m}H^{-\frac 1 2}h \, ,
\end{equation*}
hence
\begin{equation*}
    \Sigma_Y = \frac 1 m \mathbbm 1 - \frac{2}{m^2} H^{-\frac 1 2} h h^T H^{-\frac 1 2} \preccurlyeq \esp{YY^T} \, .
\end{equation*}

\begin{theoremEnd}{claim}\label{clm:sigma-y-bound}
    $\Sigma_Y \preccurlyeq \frac 1 m \mathbbm 1$ .
\end{theoremEnd}
\begin{proofEnd}
In order to prove \autoref{clm:sigma-y-bound}, we'll use the properties of
positive-(semi)definite matrices shown in \autoref{sec:psd-prop} to show that
$\frac{1}{m^2} H^{-\frac 1 2} h h^T H^{-\frac 1 2} \succcurlyeq 0$, where $H = A
A^T, A = B^TS^{-1}$ and $h = B^TS^{-1}\ket{\bs 1}$.

Since $H = A A^T$, $H \succcurlyeq 0$\footnote{We can assume $H \succ 0$ since --
in the context of IPM -- $H$ is nonsingular.} thus by
\autoref{prop:psd-inverse} $H^{-1} \succ 0$ and by \autoref{prop:psd-root}
$H^{-\frac{1}{2}} \succ 0$ Hence,
\begin{equation*}
\begin{aligned}
 H^{-\frac{1}{2}} h h^T H^{-\frac{1}{2}}
    & = (H^{-\frac{1}{2}}h)(h^TH^{-\frac{1}{2}}) \\
    & = (H^{-\frac{1}{2}}h)(H^{-\frac{1}{2}}h)^T \\
    & \succcurlyeq 0
\end{aligned}
\end{equation*}
which implies,
    $$ \Sigma_Y \preccurlyeq \frac 1 m \mathbbm 1 - \frac{1}{m^2} H^{-\frac 1 2} h h^T H^{-\frac 1 2} \preccurlyeq \displaystyle\frac 1 m \mathbbm 1$$
\end{proofEnd}
Using \autoref{th:estimate-mean} with the random variable $Y$ yields $\tilde \mu_Y$ such
that
\begin{equation}\label{eq:apply-th-y}
\| \mu_Y - \tilde \mu_Y\|_2 \leq \frac{\sqrt{n \text{tr}(\Sigma_Y)}}{T} \, .
\end{equation}
Having $\Sigma_Y  \preccurlyeq \frac 1 m \mathbbm 1$ implies
$\text{tr}(\Sigma_Y) \leq \frac{n}{m}$ since $\Sigma_Y \in \mathbb R^{n \times
n}$. Following our initial assumption -- \autoref{eq:estimate-bound} -- we want
the left side of \autoref{eq:apply-th-y} to be smaller than $\frac \varepsilon
m$, \emph{i.e.},
\begin{equation*}
    \frac{n}{\sqrt m T} \leq \frac \varepsilon m \, ,
\end{equation*}
which is achieved if and only if
\begin{equation*}
    T \geq \frac{n \sqrt m}{\varepsilon} \, .
\end{equation*}


\subsubsection{Using sparsified Hessian}
We consider we have a $\lambda$-spectral approximation $\tilde{H}$  of $H$ such
that
\begin{equation}\label{eq:approx}
\frac{1}{\lambda} H \preccurlyeq \tilde H \preccurlyeq H \, ,
\end{equation}
which we can rephrase
\begin{equation} \label{eq:approx-root}
\frac{1}{\sqrt \lambda} H^{\frac{1}{2}} \preccurlyeq \tilde H^{\frac{1}{2}} \preccurlyeq  H^{\frac{1}{2}} \, ,
\end{equation}
and we'll use another straightforward formulation, \emph{i.e.},
\begin{equation} \label{eq:approx-inverse-root}
H^{-\frac{1}{2}} \preccurlyeq \tilde H^{- \frac{1}{2}} \preccurlyeq \sqrt \lambda H^{-\frac{1}{2}} \, .
\end{equation}
In a similar manner as in \autoref{sec:initial-approach}, we define $\tilde Y =
\tilde H^{-\frac{1}{2}}X = \tilde H^{-\frac{1}{2}} B^T S^{-1}\ket{e}$ where $e
\sim \mathcal{U}[m]$. This yields
\begin{equation*}
    \begin{aligned}
        \esp{\tilde Y \tilde Y^T}
            & = \tilde H^{-\frac 1 2} \esp{XX^T}  \tilde H^{-\frac 1 2} \\
            & = \frac 1 m \tilde H^{-\frac 1 2} H \tilde H^{-\frac 1 2} \, .
    \end{aligned}
\end{equation*}
From \autoref{prop:product-of-root-psd}, we deduce
\begin{equation*}
    \frac {1}{m} \mathbbm 1\preccurlyeq \esp{YY^T} \preccurlyeq \frac{\lambda}{m} \mathbbm 1.
\end{equation*}
Tracing each part, and using \autoref{clm:sigma-y-bound} yields
\begin{equation*}
    \text{tr}(\Sigma_{\tilde{Y}}) \leq \frac{n\lambda}{m} \, .
\end{equation*}
The \autoref{th:estimate-mean} ensures the existence of $\tilde
\mu_{\tilde Y}$ such that
\begin{equation*}
\| \mu_{\tilde Y} - \tilde \mu_{\tilde Y} \|_2
    \leq \frac{\sqrt{n \text{tr}(\Sigma_{\tilde Y})}}{T}
    \leq \frac{n \sqrt \lambda}{\sqrt m T}.
\end{equation*}
Recall that we want $\| \mu_{\tilde Y} - \tilde \mu_{\tilde Y} \|_2 \leq \frac
\varepsilon m$. As such, we set
\begin{equation*}
T \geq \frac{n \sqrt{m\lambda}}{\varepsilon} \, .
\end{equation*}
Therefore, choosing $T$ as defined above yields
\begin{equation*}
\| \mu_{\tilde Y} - \tilde \mu_{\tilde Y} \|_2
    = \left\| \tilde H^{- \frac 1 2} (\mu_{X} - \tilde \mu_{X}) \right\|_2
    = \left\| \mu_{X} - \tilde \mu_{X} \right\|_{\tilde H^{-1}} \leq \frac \varepsilon m,
\end{equation*}
and consequently
\begin{equation}\label{eq:h-tilde-bound}
    \|h - \tilde h \|_{\tilde H^{-1}} \leq \varepsilon \, ,
\end{equation}
which is a quantity preserved by the initial sparsification: \autoref{eq:approx}
ensures that for all $\x \in \mathbb R^n$, the following holds :
\begin{equation*}
\sqrt{\x^T H^{-1} \x} \leq \sqrt{\x^T \tilde H^{-1} \x } \leq \sqrt{\lambda} \cdot \sqrt{\x^T H^{-1} \x} \, ,
\end{equation*}
hence, by definition of $\| \cdot \|_{H^{-1}}$, we have
\begin{equation*}
    \|h - \tilde h \|_{H^{-1}}
    \leq \|h - \tilde h \|_{\tilde H^{-1}}
    \leq \sqrt \lambda \|h - \tilde h \|_{H^{-1}} \, .
\end{equation*}
Note that the above inequality would have hold if we were doing $T =
\frac{n\sqrt m}{\varepsilon}$ queries to compute the approximate with respect to
both $H^{-1}$ and $\tilde H^{-1}$. However, to compute $\tilde h$ and bound it
with respect to $\| \cdot \|_{\tilde H^{-1}}$, we do $\sqrt \lambda T$ queries,
and as such, \autoref{eq:h-tilde-bound} effectively holds.


\subsubsection{Application within Newton's step -- Measure of progress}
We consider we have an approximate of the gradient $\nabla \Phi$, $\tilde \nabla
\Phi$, such that for all $\x \in \mathbb R^n$, it holds that
\begin{equation}\label{eq:approx-gradient-bound}
    \left\|\nabla \Phi (\x) - \tilde \nabla \Phi (\x) \right\|_{H^{-1}} \leq \varepsilon \, ,
\end{equation}
where $H$ is the current (in terms of the Newton's steps) Hessian matrix. The
above inequality can be directly used to bound the convergence of Newton's
method. We first recall that
\begin{equation*}
   \begin{aligned}
    x' = x - H^{-1}\nabla\Phi(x) \, ; \\
    y' = x - \tilde H^{-1}\tilde \nabla \Phi(x)\, ,
   \end{aligned}
\end{equation*}
and we define
\begin{equation*}
    \hat y = x - H^{-1} \tilde \nabla \Phi(x) \, ,
\end{equation*}
where we only use the approximate gradient, the Hessian is the actual one. The
measure of progress can be expressed as follows, using the triangular inequality
\begin{equation*}
    \| y'- x^* \|_H \leq \| y'- \hat y \|_H + \| \hat y - x' \|_H + \| x' - x^* \|_H \, ,
\end{equation*}
where we'll bound each of the three terms. The first term the least
straightforward to bound.
\begin{equation}\label{eq:bound-first-term}
    \begin{aligned}
        \| y' - \hat y \|_H
            & = \|\tilde H^{-1}\tilde \nabla \Phi(x) - H^{-1}\tilde \nabla \Phi(x) \|_H \\
            & = \| H^{\frac{1}{2}}\tilde H^{-1}\tilde \nabla \Phi(x) - H^{\frac{1}{2}}H^{-1}\tilde \nabla \Phi(x) \|_2 \\
            & = \| H^{\frac{1}{2}}\tilde H^{-1}H^{\frac{1}{2}}(H^{-\frac{1}{2}}\tilde \nabla \Phi(x)) \\
            & \qquad - H^{\frac{1}{2}}H^{-1}H^{\frac{1}{2}}(H^{-\frac{1}{2}}\tilde \nabla \Phi(x)) \|_2 \\
            & = \|(H^{\frac{1}{2}}\tilde H^{-1}H^{\frac{1}{2}} - \mathbbm 1) (H^{-\frac{1}{2}}\tilde \nabla \Phi(x))\|_2 \\
            & \leq (\lambda - 1) \left(\|x - x^*\|_H - \varepsilon\right) \, .
    \end{aligned}
\end{equation}
% \hft{Explain the steps}

The first step is the straight application of \autoref{prop:change-norm}. The
second step is a multiplication by $H^{\frac 1 2}H^{-\frac 1 2} = \mathbbm 1$ in
order to factorize and simplify in step 3. The last step is obtained as follows:
we use triangular inequality to separate it into a product of norms, and on the
one hand we have
\begin{equation*}
    \begin{aligned}
        \| H^{-\frac{1}{2}}\tilde \nabla \Phi(x)\|_2
        & \leq \| H^{-\frac{1}{2}} \nabla \Phi(x)\|_2 + \varepsilon\\
        & = \| H^{-1} \nabla \Phi(x)\|_H + \varepsilon \\
        & = \| x' - x \|_H + \varepsilon \\
        & \leq \| x - x^*\|_H + \varepsilon
   \end{aligned}
\end{equation*}
assuming that Newton's method does not overshoot. And on the other, hand we use
\autoref{prop:product-of-root-psd} together with \autoref{eq:approx} to obtain
\begin{equation*}
    \|H^{\frac{1}{2}}\tilde H^{-1}H^{\frac{1}{2}} - \mathbbm 1 \|_H \leq \lambda - 1 \, .
\end{equation*}
For the second term, from \autoref{eq:approx-gradient-bound} we have
\begin{equation}\label{eq:bound-second-term}
   \begin{aligned}
    \| \hat y - x' \|_H
        & = \|H^{-1}\nabla\Phi(x) - H^{-1}\tilde\nabla\Phi(x) \|_H \\
        & = \|\nabla\Phi(x) - \tilde\nabla\Phi(x) \|_{H^{-1}} \\
        & \leq \varepsilon
   \end{aligned}
\end{equation}
The last one is bounded by the convergence rate of the exact Newton's method
(see \emph{e.g.,} \cite[p. 43]{numerical_2006}); such that
\begin{equation}\label{eq:bound-third-term}
    \| x' - x^* \|_H \leq \frac{1}{2} \| x - x^*\|^2_H \, .
\end{equation}
Gathering \autoref{eq:bound-first-term}, \autoref{eq:bound-second-term}, and
\autoref{eq:bound-third-term}, we obtain the following measure of progress.
\begin{equation*}
    \begin{aligned}
        \| y' - x^* \|_H
            & \leq (\lambda - 1) \left(\|x - x^*\|_H - \varepsilon\right) + \varepsilon + \frac{1}{2} \| x - x^*\|_H \\
            & \leq (\lambda - \frac{1}{2})\|x - x^*\|_H + (2 - \lambda) \varepsilon \, .
    \end{aligned}
\end{equation*}

In order to have, let's say, $\lambda - \frac{1}{2} = \frac{9}{10}$, we set
$\lambda = \frac{14}{10}$. Recall that $\lambda$ is the approximation factor for
the sparsification of the Hessian, and from
\autoref{eq:bound-opt-approx-hessian}, we wanted $1 \leq \lambda < \frac{3}{2}
$, which is maintained here. In order to end up with a \emph{clean} expression
for the measure of progress, we set $\varepsilon \leftarrow \frac{3}{5}
\varepsilon$, and as such, we have
\begin{equation*}
        \| y' - x^* \|_H \leq \frac{9}{10}\|x - x^*\|_H + \varepsilon \, .
\end{equation*}

With this measure of progress, we can prove that we converge towards the
minimizer $x^*$ when we consider the approximate procedure. More formally we
show that $y^*$ is in an $O(\varepsilon)$-ball around $x^*$.

\subsection{Convergence of the Approximate Newton's method}

See \autoref{sec:recursive-sequence} for the method for finding the
$k$-th term of a recursive sequence. We first recall that we have
the following bound
\begin{equation}\label{eq:upper-bound-a-1}
    \| y_1 - x^* \|_H \leq \frac{9}{10} \| y_0 - x^* \|_H + \varepsilon \, ,
\end{equation}
as such we can define the following recursive sequence :
\begin{equation}\label{eq:definition-f-ipm}
    \begin{cases}
        a_0 = \| y_0 - x^* \|_H \\
        a_{k+1} = \frac{9}{10} a_k + \varepsilon
    \end{cases} \, ,
\end{equation}
It is easy to see that \autoref{eq:upper-bound-a-1} shows an upper bound on
$a_1$. For the sake of simplicity, we can set $a_1 = \| y_1 - x^* \|_H$.
\begin{theoremEnd}{claim}
    With \autoref{clm:a-k+1-value}, we show that
    \begin{equation*}
        a_{k+1} = (\frac{9}{10})^{k+1} a_0 + O(\varepsilon) \, .
    \end{equation*}
    proving that our procedure converges.
\end{theoremEnd}
\begin{proofEnd} Here $p = \frac{9}{10}$ and $q = \varepsilon$.
    \begin{equation}\label{eq:valueof-ak-1}
        \begin{aligned}
            a_{k+1}
                & = (a_1 - \frac{q}{1-p})p^k + \frac{q}{1-p} \\
                & = (a_1 - 10 \varepsilon)(\frac{9}{10})^k + 10 \varepsilon \\
                & = (\frac{9}{10})^k a_1 - (\frac{9}{10})^k 10 \varepsilon + 10 \varepsilon \\
                & \leq (\frac{9}{10})^k (\frac{9}{10} a_0 + \varepsilon) - (\frac{9}{10})^k 10 \varepsilon + 10 \varepsilon \\
                & = (\frac{9}{10})^k \frac{9}{10} a_0 + (\frac{9}{10})^k \varepsilon - (\frac{9}{10})^k 10 \varepsilon + 10 \varepsilon \\
                & = (\frac{9}{10})^{k+1} a_0 + 10 \varepsilon \left(\frac{1}{10}(\frac{9}{10})^k - (\frac{9}{10})^k +1\right) \\
                & = (\frac{9}{10})^{k+1} a_0 + 10 \varepsilon \left((\frac{9}{10})^k (\frac{1}{10} - 1) +1\right) \\
                & = (\frac{9}{10})^{k+1} a_0 + 10 \varepsilon \left(- (\frac{9}{10})^k \frac{9}{10} +1\right) \\
                & = (\frac{9}{10})^{k+1} a_0 + 10 \varepsilon \left(1 - (\frac{9}{10})^{k+1}\right) \\
        \end{aligned}
    \end{equation}
\end{proofEnd}
By reducing \autoref{prop:f-convergence-mapping} to our case where is $f$
defined in \autoref{eq:definition-f-ipm} we have $p = \frac{9}{10}$, hence, $f$
is a contraction mapping. \autoref{thm:banach-fixed-point} states that $f$ has
an unique fixed point $\bar a$. Since
\begin{equation*}
    \lim_{k \rightarrow \infty} (\frac{9}{10})^{k} = 0 \, ,
\end{equation*}
The fixed point of $f$ is
\begin{equation*}
    \begin{aligned}
        \bar a
            & = \lim_{k \rightarrow \infty} a_{k} \\
            & = \lim_{k \rightarrow \infty} \left( (\frac{9}{10})^{k} a_0 + 10 \varepsilon \left(1 - (\frac{9}{10})^{k}\right) \right) \\
            & = 10\varepsilon
    \end{aligned}
\end{equation*}

It is important to stress that $y^*$ such that $\tilde \nabla \Phi(y^*) = 0$ is
never computed. Typically there is not a finite rank $r$ such that $a_r = \bar
a$. However, $a_r$ can be arbitrarily close to $\bar a$ for a finite $r$. We
choose $r$ such that, for instance, $a_r \leq \frac{11}{10} \bar a$, which
implies that for all $k > r$,
\begin{equation*}
    \|y_k - x^*\| \leq 11\varepsilon \, .
\end{equation*}
Thus, for all such $k$, Newton's steps will keep on oscillating inside the
$11\varepsilon$-ball around $x^*$. As such, as soon as we obtain a $y_k$ that is
within the $11\varepsilon$-ball around $x^*$, it is wise to stop the procedure
since doing one more step will produce a $y_{k+1}$ for which we have no way to
say whether it will be closer to $x^*$ than $y_k$ was. Using
\autoref{eq:valueof-ak-1}, we can easily show that
\begin{equation*}
    r \leq \log_p(\varepsilon^{-1}) \, ,
\end{equation*}
with $p = \frac{9}{10}$. Consequently, the stop-condition of \autoref{alg:ipm}
becomes $\|y_k - x^*\| \leq 11 \varepsilon$, and we can expect that to happen
after $\log_p(\varepsilon^{-1})$ Newton's steps.