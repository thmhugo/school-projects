\appendix
\subsection{Johnson-Lindenstrauss lemma}
\label{ap:jllemma}
The Johnson-Lindenstrauss lemma \cite{jl_transfrom} is a result stating that it is possible to embed
with \textit{low-distortion}, \textit{i.e., } by preserving the pairwise
distance between any pair of points within a factor $(1 \pm \varepsilon)$,
points from high-dimensional into low-dimensional Euclidean space. It is
appropriate to consider the distributional statement of the
Johnson-Lindenstrauss lemma (DJL), in order to address it not in terms of
parwise distances but with respect to the resulting distortion of the mapping.

\begin{lemma}[Distributional Johnson-Lindenstrauss lemma]\label{jl-lemma}
For any $0 < \varepsilon < 1$, $\delta < 1/2$ and $d \in \mathbb N$, there
exists a distribution over $\displaystyle \mathbb {R} ^{k\times d}$ from which
the matrix $\Pi$ is drawn such that for $k =
O\big(\varepsilon^{-2}\log(\delta^{-1})\big)$ and any vector $\x \in \mathbb
R^d$, the following claim holds.
$$
\mathbb P \Big(\left | \| \Pi \x \|_2^2 - \|\x\|_2^2 \right | > \varepsilon \cdot \| \x \| _2^2
\Big)<\delta \, .
$$
\end{lemma}

This lemma states that there exists a matrix $\Pi$ that reduces the column space
of the initial space, and with probability $\geq 1-\delta$ the norm of a given
vector in the reduced space is a good multiplicative approximation of its norm
original norm.

\subsection{$k$-independent hash functions}\label{ap:def-k-independent} It is
important to stress that the results presented in this subsection are purely
classical.
\begin{definition}[$k$-independent hashing functions]\label{def:k-independent-hashing}
    Let $\mathcal U$ be the set of keys. A family $\mathcal H = \big\{ h : \mathcal
    U \rightarrow [m]\big\}$ is said to be $k$-independent if for all keys $x_1,
    \cdots, x_k$ in $\mathcal U$ pairwise distinct and for all values $v_1, \cdots,
    v_k$ in $[m]$,
    \begin{equation*}
        \big| \{ h \in \mathcal H \; ;\; h(x_1)=v_1, \cdots,  h(x_k)=v_k \} \big| =
        \frac{|\mathcal H |}{m^k} \text{ ,}
    \end{equation*}
    in other words, by providing $\mathcal H$ with the uniform probability, for any
    $h\in \mathcal H$
    \begin{equation*}
        \mathbbm{P}\Big(h(x_1)=v_1, \cdots,  h(x_k)=v_k \Big) = \frac{1}{m^k} \text{ .}
    \end{equation*}
\end{definition}
Using the result of \cite{christiani_independence_2015} on $k$-wise independent
hash functions, rephrased by \cite{apers_quantum_2020} yields the following
theorem.
\begin{theorem}\label{thm:k-indep-data-structure}
    It is possible to construct in time $\tilde O(k)$ a data structure of size
    $\tilde O(k)$ that allows to simulate queries to a k-independent hash
    function in $\tilde O(1)$ time per query.
\end{theorem}

\subsection{Interior Points Method}\label{ap:ipm}
In the IPM framework, the following linear program is considered, which
corresponds to the minimum cost flow problem.

\begin{equation}
    \begin{array}{ll@{}l}
    \text{minimize}  & c^T f \\
    \text{subject to}& B^Tf = d   \\
                     & f \geq 0
    \end{array} \text{.}
\end{equation}

By using the Barrier formulation, one can put the constraints in the ojective
and enable an iterative approximation of the optimal solution by studying the
objective function.

\begin{equation}
  \begin{array}{ll@{}l}
    \text{minimize } & \Phi_{\mu}(x) = - \frac{d^{T}x}{\mu} - \underbrace{\displaystyle\sum^{n}_{i=1} \ln (\underbrace{c - Bx}_{\text{slack}})}_{\text{log barrier}} \\
  \end{array}
\end{equation}

The log term provides a penality when the slack is close to 0 (so when the
current solution is close to the edges of the polytope). The slack should always
be positive, i.e.\ $c - Bx \geq 0$ -- this is a difficult condition to maintain.
The $\mu$ term should be very small for a good approximation of the optimal
solution, e.g.{} $\mu = n^{-100}$, but in the first steps of IPM we start with a
large value of $\mu$ in such a way that the only significant term in the
objective function is the log barrier. We define $$X_{\mu}^{*} =
\displaystyle\argmin_{x} \Phi_{\mu} \lvert x \rvert \text{,}$$ such that the
$$\text{central path} = \left\{ X_{\mu}^{*}: \mu \in [0,\infty]\right\} \text{.}$$

Iterating over close values of $\mu$ is efficient: i.e.\ it is efficient to
compute $X_{\mu_{l}}^{*}$ from $X_{\mu_{l-1}}^{*}$ and $X_{\mu_{l+1}}^{*}$ from
$X_{\mu_{l}}^{*}$, each iteration requires $\sqrt{m}$ iterations with the Newton's
method.

We consider the gradient
$$\nabla \Phi_{\mu}(X) = - \frac{d}{\mu} + \frac{B^T}{\vec s},$$ which, when at
optimal, gives
$$B^{T}\frac{\mu}{\vec s} = d.$$
We have $$\nabla^{2} \Phi_{\mu}(X) = B^{T}S^{-2}B,$$ where $S =
\mathrm{diag}(\vec s)$. By applying the Newton's method, we obtain
$$X'=X-(\underbrace{B^{T}S^{-2} B}_{\text{the Hessian }H})^{-1}\nabla
\Phi_{\mu}(x) \, .$$
% The new slacks can be expressed as
% $$
% \begin{aligned}
%   \vec s \, '&= c - Bx' \\
%     &= \vec s + B(x - x') \\
%     &= B H^{-1} \nabla \Phi_{\mu}(X)
% \end{aligned}
% $$

The expression $H^{-1} \nabla \Phi_{\mu}(X)$ corresponds to a $\Delta x$ that
satisfies $B^{T} S^{-2} B \Delta x = \nabla \Phi_{\mu}(x)$ and is referred to as
the vector potential. Thus it is not necessary to inverse $H$, one could solve
$H \Delta x = \nabla \Phi_{\mu}(x)$ to find $\Delta x$.


A summary of the algorithm is given here.
\begin{algorithm}[H]
    \caption{Interior points method}
    \label{alg:ipm}
    \begin{algorithmic}[1]
        \Require $B \in \mathbb R^{n\times m}, \varepsilon > 0, \mu \gg 1,  \delta = \frac{1}{10\sqrt{m}}, x_0$ feasible solution
        % \State $x_\mu^* \leftarrow \argmin \Phi_\mu(x)$ \Comment{Obtained with Newton Method}
        \If {$\mu = \varepsilon$}
            \State exit
        \EndIf
        \State $k \leftarrow 0$
        \While{$\nabla \Phi_\mu(x_k) \neq 0$} \Comment{Newton's method}
            \State $x_{k+1} \leftarrow x_k - \left[B^TS^{-2}B\right]^{-1}\nabla \Phi_\mu(x_k)$
            \State $k \leftarrow k+1$
        % \State $\vec s = $
        \EndWhile \Comment{here $x = \argmin \Phi_\mu$}
        \State $\mu \leftarrow \frac{\mu}{1 + \delta}$  \Comment{Advancing down the central path}
        \State iterate once again with new values of $x_k$ and $\mu$
    \end{algorithmic}
\end{algorithm}


Complexity-wise, it requires to solve $\tilde O(\sqrt{m})$ linear systems to
move from $x_\mu^*$ to $x_\frac{\mu}{1 + \delta}^*$ (\emph{i.e.,} finding
$\argmin \Phi_\mu$), hence a total of $\tilde O(\frac{\sqrt m}{\varepsilon})$ to
get to $\mu = \varepsilon$.

\subsection{Some properties of PSD matrices}\label{sec:psd-prop}
We first define positive-semidefinite (PSD) matrices, and then expose some
properties that we use along the report.
\begin{definition}[PSD matrix]
An $n \times n$ symmetric real matrix $M$ is said to be positive-semidefinite if,
for all nonzero vector $\x \in \mathbb R^n$,
\begin{equation*}
    \x^T M \x \geq 0 \, .
\end{equation*}
And, equivalently, $M$ is said to be positive-semidefinite if all its
eigenvalues are nonnegative.
\end{definition}
If $M$ is PSD, then we write $M \succcurlyeq 0$.
\begin{proposition}[Inverse of PSD matrix]\label{prop:psd-inverse}
    $$H  \succ 0 \Rightarrow H^{-1} \succ 0$$
\end{proposition}
\begin{proof}
    If $H \succ 0$, then $H$ is invertible. Let $y := Hx $, thus for all $x$
    \begin{equation*}
    y^T H^{-1}y = x^T H^TH^{-1}Hx = x^T H^T x = x^THx > 0 \, .
    \end{equation*}
\end{proof}

\begin{proposition}[Root of PSD matrix]\label{prop:psd-root}
    $$H^{-1}  \succcurlyeq 0 \Rightarrow H^{-\frac{1}{2}} \succcurlyeq 0$$
\end{proposition}
\begin{proof}
    Diagonalize $H^{-1}$, obtain $H^{-\frac{1}{2}}$ by taking the root of the
    eigenvalues. Since the function $f : x \mapsto \sqrt x$ is a positive
    function, all eigenvalues of $H^{-\frac{1}{2}}$ are nonnegative.
\end{proof}

\begin{proposition}[PSD matrix product]\label{prop:product-of-root-psd}
    $$A \preccurlyeq B \Rightarrow \mathbbm 1 \preccurlyeq A^{-\frac 1 2}BA^{-\frac 1 2} \, . $$
\end{proposition}
\begin{proof}
By definition of Loewner order, we say that $A \preccurlyeq B $ if and only if $
0 \preccurlyeq B - A$, \emph{i.e.,} for all vector $x$, it holds that
\begin{equation}\label{eq:def-psd-matrix-order}
    x^T(B-A)x \geq 0  \, .
\end{equation}
Multiplying each element of \autoref{eq:def-psd-matrix-order} by $A^{-\frac 1 2}$
gives
\begin{equation*}
    (A^{-\frac 1 2}x)^T(A^{-\frac 1 2}BA^{-\frac 1 2}-\mathbbm 1)(A^{-\frac 1 2}x) \geq 0 \, .
\end{equation*}
where we can define $y := A^{-\frac 1 2}x$, and as such it holds that for all $y$,
\begin{equation*}
    y^T(A^{-\frac 1 2}BA^{-\frac 1 2}-\mathbbm 1)y \geq 0 \, ,
\end{equation*}
that is,
\begin{equation*}
    \mathbbm 1  \preccurlyeq A^{-\frac 1 2}BA^{-\frac 1 2} \, .
\end{equation*}
\end{proof}
\begin{proposition}[Changing the norm]\label{prop:change-norm}
$\forall \x \in \mathbb R^{n}, H\succcurlyeq 0 \in \mathbb R^{n\times n}\, ,$

$$\| \x \|_H = \| H^{\frac{1}{2}}\x\|_2 $$
\end{proposition}
\begin{proof}
    By definition of the weighted norm with respect to a matrix $H$
    (\autoref{eq:def-h-norm}), and since $H$ is assumed PSD,
    \begin{equation*}
        \begin{aligned}
            \| \x \|_H & = \sqrt{\x^T H \x} \\
                & = \sqrt{\x^T H^{\frac{1}{2}} H^{\frac{1}{2}} \x} \\
                & = \sqrt{\x^T (H^{\frac{1}{2}})^T H^{\frac{1}{2}} \x} \\
                & = \sqrt{(H^{\frac{1}{2}}\x)^T H^{\frac{1}{2}} \x} \\
                & = \| H^{\frac{1}{2}} \x \|_2
        \end{aligned}
    \end{equation*}

\end{proof}

\subsection{Recursive sequences}\label{sec:recursive-sequence}
%
A recursive sequence $(a_k)_{k \in \mathbb N}$ is defined by as follows
\begin{equation} \label{eq:gen-rec-seq}
    \begin{cases}
        a_0 \in \mathbb R \\
        a_{k+1} = p a_k + q = f(a_k), & p, q \in \mathbb R \,
    \end{cases} \, .
\end{equation}

\begin{definition}[Fixed point of a sequence]
A fixed point of a sequence is an $\bar a$ such that
\begin{equation*}
    f(\bar a) = \bar a \, .
\end{equation*}
\end{definition}

\begin{definition}[Contraction mapping]\label{def:contraction-mapping}
   A function $f$ is said to be a contraction if for all $x, y$ and a constant
   $0 \leq K < 1$,
   \begin{equation*}
    \| f(x)-f(y) \| \leq K \| x - y \| \, .
   \end{equation*}
\end{definition}

\begin{theorem}[Banach fixed-point theorem, informal]\label{thm:banach-fixed-point}
If $f: \mathbb R \rightarrow \mathbb R$ is a contraction mapping, then $f$
admits an unique fixed point $\bar a$. Furthermore,
\begin{equation*}
    \lim_{k\rightarrow \infty} a_k = \bar a \, .
\end{equation*}
\end{theorem}

Given the definition of $f$, it is possible to directly evaluate $a_{k+1}$.

\begin{proposition}\label{clm:a-k+1-value}
$\forall k \geq 0,  \quad a_{k+1} = (a_1 - \frac{q}{1-p})p^k + \frac{q}{1-p}$.
\end{proposition}
\begin{proof} By recurrence on $k$. \\
    \textbf{Base case:} $(a_1 - \frac{q}{1-p})p^0 + \frac{q}{1-p} = a_1$ \\
    \textbf{Inductive case:} We assume $$p_k : a_{k+1} = (a_1 - \frac{q}{1-p})p^k +
    \frac{q}{1-p}$$ true, and we show that $$p_{k+1} : a_{k+2} = (a_1 -
    \frac{q}{1-p})p^{k+1} + \frac{q}{1-p}$$ is also true.
    \begin{equation*}
        \begin{aligned}
            a_{k+2}
            & = p a_{k+1} + q \\
            & = p \left((a_1 - \frac{q}{1-p})p^k + \frac{q}{1-p}\right) + q \\
            & = (a_1 - \frac{q}{1-p})p^{k+1} + \frac{pq}{1-p} + \frac{q(1-p)}{1-p} \\
            & = (a_1 - \frac{q}{1-p})p^{k+1} + \frac{pq - pq + q}{1-p} \\
            & = (a_1 - \frac{q}{1-p})p^{k+1} + \frac{q}{1-p} \\
        \end{aligned}
    \end{equation*}
\textbf{Conclusion:} $p_k$ implies $p_{k+1}$ for all $k$, which proves
\autoref{clm:a-k+1-value}.
\end{proof}

\begin{proposition}\label{prop:f-convergence-mapping}
    $f$ as defined in \autoref{eq:gen-rec-seq} is a contraction mapping if $0
    \leq |p| < 1$.
\end{proposition}
\begin{proof}
    For all $x,y$ we have
    \begin{equation*}
        \begin{aligned}
            \| f(x) - f(y) \| = \| p(x-y) \| \leq |p| \cdot \| x-y \| \, .
        \end{aligned}
    \end{equation*}
    Thus, by \autoref{def:contraction-mapping} $f$ is a contraction mapping if
    and only if $0 \leq |p| < 1$.
\end{proof}

\subsection{Proofs}
\printProofs