\subsection{Convergence of the Approximate Newton's method}

See \autoref{sec:recursive-sequence} for the method for finding the
$k$-th term of a recursive sequence. We first recall that we have
the following bound
\begin{equation}\label{eq:upper-bound-a-1}
    \| y_1 - x^* \|_H \leq \frac{9}{10} \| y_0 - x^* \|_H + \varepsilon \, ,
\end{equation}
as such we can define the following recursive sequence :
\begin{equation}\label{eq:definition-f-ipm}
    \begin{cases}
        a_0 = \| y_0 - x^* \|_H \\
        a_{k+1} = \frac{9}{10} a_k + \varepsilon
    \end{cases} \, ,
\end{equation}
It is easy to see that \autoref{eq:upper-bound-a-1} shows an upper bound on
$a_1$. For the sake of simplicity, we can set $a_1 = \| y_1 - x^* \|_H$.
\begin{theoremEnd}{claim}
    With \autoref{clm:a-k+1-value}, we show that
    \begin{equation*}
        a_{k+1} = (\frac{9}{10})^{k+1} a_0 + O(\varepsilon) \, .
    \end{equation*}
    proving that our procedure converges.
\end{theoremEnd}
\begin{proofEnd} Here $p = \frac{9}{10}$ and $q = \varepsilon$.
    \begin{equation}\label{eq:valueof-ak-1}
        \begin{aligned}
            a_{k+1}
                & = (a_1 - \frac{q}{1-p})p^k + \frac{q}{1-p} \\
                & = (a_1 - 10 \varepsilon)(\frac{9}{10})^k + 10 \varepsilon \\
                & = (\frac{9}{10})^k a_1 - (\frac{9}{10})^k 10 \varepsilon + 10 \varepsilon \\
                & \leq (\frac{9}{10})^k (\frac{9}{10} a_0 + \varepsilon) - (\frac{9}{10})^k 10 \varepsilon + 10 \varepsilon \\
                & = (\frac{9}{10})^k \frac{9}{10} a_0 + (\frac{9}{10})^k \varepsilon - (\frac{9}{10})^k 10 \varepsilon + 10 \varepsilon \\
                & = (\frac{9}{10})^{k+1} a_0 + 10 \varepsilon \left(\frac{1}{10}(\frac{9}{10})^k - (\frac{9}{10})^k +1\right) \\
                & = (\frac{9}{10})^{k+1} a_0 + 10 \varepsilon \left((\frac{9}{10})^k (\frac{1}{10} - 1) +1\right) \\
                & = (\frac{9}{10})^{k+1} a_0 + 10 \varepsilon \left(- (\frac{9}{10})^k \frac{9}{10} +1\right) \\
                & = (\frac{9}{10})^{k+1} a_0 + 10 \varepsilon \left(1 - (\frac{9}{10})^{k+1}\right) \\
        \end{aligned}
    \end{equation}
\end{proofEnd}
By reducing \autoref{prop:f-convergence-mapping} to our case where is $f$
defined in \autoref{eq:definition-f-ipm} we have $p = \frac{9}{10}$, hence, $f$
is a contraction mapping. \autoref{thm:banach-fixed-point} states that $f$ has
an unique fixed point $\bar a$. Since
\begin{equation*}
    \lim_{k \rightarrow \infty} (\frac{9}{10})^{k} = 0 \, ,
\end{equation*}
The fixed point of $f$ is
\begin{equation*}
    \begin{aligned}
        \bar a
            & = \lim_{k \rightarrow \infty} a_{k} \\
            & = \lim_{k \rightarrow \infty} \left( (\frac{9}{10})^{k} a_0 + 10 \varepsilon \left(1 - (\frac{9}{10})^{k}\right) \right) \\
            & = 10\varepsilon
    \end{aligned}
\end{equation*}

It is important to stress that $y^*$ such that $\tilde \nabla \Phi(y^*) = 0$ is
never computed. Typically there is not a finite rank $r$ such that $a_r = \bar
a$. However, $a_r$ can be arbitrarily close to $\bar a$ for a finite $r$. We
choose $r$ such that, for instance, $a_r \leq \frac{11}{10} \bar a$, which
implies that for all $k > r$,
\begin{equation*}
    \|y_k - x^*\| \leq 11\varepsilon \, .
\end{equation*}
Thus, for all such $k$, Newton's steps will keep on oscillating inside the
$11\varepsilon$-ball around $x^*$. As such, as soon as we obtain a $y_k$ that is
within the $11\varepsilon$-ball around $x^*$, it is wise to stop the procedure
since doing one more step will produce a $y_{k+1}$ for which we have no way to
say whether it will be closer to $x^*$ than $y_k$ was. Using
\autoref{eq:valueof-ak-1}, we can easily show that
\begin{equation*}
    r \leq \log_p(\varepsilon^{-1}) \, ,
\end{equation*}
with $p = \frac{9}{10}$. Consequently, the stop-condition of \autoref{alg:ipm}
becomes $\|y_k - x^*\| \leq 11 \varepsilon$, and we can expect that to happen
after $\log_p(\varepsilon^{-1})$ Newton's steps.