\documentclass[10pt, letterpaper, twocolumn]{article}
\usepackage[T1]{fontenc}
\usepackage[utf8]{inputenc}
\usepackage{ geometry }
\geometry{margin=1.5cm, top=2cm, bottom=2cm}
\usepackage{ amssymb }
\usepackage{ bbm }
\usepackage{ braket }
\usepackage{ mathrsfs }
\usepackage{ geometry }
\usepackage{ amsmath }
%\usepackage[alphabetic]{ amsrefs }
\usepackage{ hyperref }
\usepackage{ enumitem }
\usepackage{ graphicx }
\setlength{\columnsep}{0.8cm}
\usepackage{ blindtext }
\usepackage{ caption }
\usepackage{ bookmark }
\usepackage{ float }
\usepackage{ stfloats }
\usepackage{ pgf }
\usepackage{ tikz }
\usepackage{ pgfplots }
\usetikzlibrary{graphs,graphs.standard,quotes,arrows.meta}
\usepackage{ authblk }
\usepackage{ sectsty } % section/subection font size
\usepackage{ algorithm }
\usepackage[noend]{ algpseudocode }
\usepackage{ amssymb }
\usepackage{ bbm }
\usepackage{ braket }
\usepackage{ mathrsfs }
\usepackage{comment}
\usepackage{amsthm}
\usepackage{xcolor}
\usepackage{ geometry }
\usepackage{ amsmath }
%\usepackage[alphabetic]{amsrefs}
\usepackage{ hyperref }
\usepackage{ bbm }
\usepackage{ diagbox }
\usepackage{ multirow }
\usepackage[super]{nth}
\usepackage{authblk}
\usepackage{cleveref}
\usepackage{proof-at-the-end}
\usepackage[english]{babel}

\usepackage[style=alphabetic,natbib=true,backend=bibtex]{biblatex}
\usepackage{refcount}

\AtEveryBibitem{
    \clearfield{urlyear}
    \clearfield{urlmonth}
    \clearfield{url}
    % \clearfield{doi}
    \clearfield{note}
    \clearfield{issn}
}

\addbibresource{misc/bibliography.bib} %Imports bibliography file

\usetikzlibrary{shapes.geometric}
\usetikzlibrary{decorations}
\usetikzlibrary{arrows.meta}

\usepackage{ fancyhdr }
\pagestyle{fancy}
\fancyhf{}
\fancyhead[LOE]{\leftmark}
\fancyfoot[COE]{\thepage}

\fancypagestyle{firststyle}
{
   \fancyhf{}
   \fancyfoot[C]{\thepage}
   \renewcommand{\headrulewidth}{0pt} % removes horizontal header line
   \newgeometry{margin=1.5cm, bottom=2cm}
}

\theoremstyle{plain}
\newtheorem{theorem}{Theorem}[section]
\newtheorem{corollary}[theorem]{Corollary}
\newtheorem{lemma}[theorem]{Lemma}
\newtheorem{proposition}[theorem]{Proposition}

\theoremstyle{definition}
\newtheorem{definition}[theorem]{Definition}
\newtheorem{example}[theorem]{Example}
\newtheorem{procedure}[theorem]{Procedure}
\newtheorem{property}[theorem]{Property}
\newtheorem{notation}[theorem]{Notation}
\newtheorem{convention}[theorem]{Convention}
\newtheorem{fact}{Fact}
\newtheorem{interpretation}[theorem]{Interpretation}
\newtheorem{remark}[theorem]{Remark}
% \newtheorem{question}[theorem]{Question}
\newtheorem{note}[theorem]{Note}
\newtheorem{claim}{Claim}
\newtheorem{question}{Question}
\theoremstyle{plain}
\newtheorem{assertion}{Assertion}[question]

\newcommand{\sref}[2]{\hyperref[#2]{#1 \ref{#2}}}



\renewcommand{\epsilon}{\varepsilon}
\newcommand{\x}{\chi}

% Figures in muticols.
% Captions with \captionof{figure}{}
\newenvironment{Figure}
  {\par\medskip\noindent\minipage{\linewidth}}
  {\endminipage\par\medskip}

% Flush left the title and authors.
\makeatletter
\renewcommand{\maketitle}{\bgroup\setlength{\parindent}{0pt}
\begin{flushleft}
  \begin{center}\Large{\centering\textbf{\@title}}\end{center}
  \normalsize\@author
  \vspace{.5cm}
  \begin{center}\@date\end{center}

\end{flushleft}\egroup
}
\makeatother

\renewcommand\Authands{ and }

\sectionfont{\fontsize{12}{12}\selectfont}
\subsectionfont{\fontsize{11}{11}\selectfont}


\newcommand{\hpa}[1]{\textcolor{teal}{#1}}
\newcommand{\ft}[1]{\textcolor{blue}{#1}}
\newcommand{\hft}[1]{\textcolor{purple}{#1}}

%%%%%%%%%%%
% Autoref %
%%%%%%%%%%%

% Rename section, subsection, subsubsection into Section
\addto\extrasenglish{%
  \def\sectionautorefname{Section}
  \def\subsectionautorefname{Section}
  \def\subsubsectionautorefname{Section}
  \def\algorithmautorefname{Algorithm}
}

\renewenvironment{abstract}
{\small
  \begin{center}
    \bfseries Abstract\vspace{-.5em}\vspace{0pt}
  \end{center}
  \list{}{
    \setlength{\leftmargin}{0mm}
    \setlength{\rightmargin}{\leftmargin}
  }
\item\relax}
{\endlist}


\usepackage[small]{titlesec}

\DeclareFieldFormat*{citetitle}{\textit{#1}}
\DeclareCiteCommand{\citejournal}
  {\usebibmacro{prenote}}
  {\usebibmacro{citeindex}%
    \usebibmacro{journal}}
  {\multicitedelim}
  {\usebibmacro{postnote}}


\newcommand{\<}{\leftarrow}
\renewcommand{\>}{\rightarrow}

\newcommand{\bs}{\boldsymbol}
\newcommand{\A}{\boldsymbol A}
\newcommand{\B}{\boldsymbol B}

\newcommand{\esp}[1]{\mathbb E \left[ #1 \right]} % Command for E[..]

% \usepackage{titlesec}

% \renewcommand{\thesection}{\Roman{section}}
% \renewcommand{\thesubsection}{\thesection.\Roman{subsection}}

% % \titleformat*{\section}{\large\bfseries}
% \titleformat{\section}
%   {\large\scshape\filcenter}{\thesection}{1em}{}
% \titleformat{\subsection}
%   {\normalsize\scshape\filcenter}{\thesubsection}{1em}{.}
% \titleformat{\subsubsection}
%   {\normalsize\scshape\filcenter}{\thesubsubsection}{1em}{}


\renewcommand{\definitionautorefname}{Definition}
\renewcommand{\propositionautorefname}{Proposition}


\title{Quantum speedup for matrix approximation via uniform sampling}
\author[]{Hugo Thomas\footnote{\texttt{hugo.thomas.3@etu.sorbonne-universite.fr}} --
Sorbonne Université, Master 1 in Quantum Information, Paris}
% \affil[]{}

\date{\today}

\begin{document}

\maketitle

\begin{abstract}
    This document is the report of the three-month internship done under the
    supervision of Simon Apers\footnote{\texttt{apers@irif.fr}} at the Institut de Recherche en Informatique
    Fondamentale (IRIF), Paris.
    % The first four sections are the result of a paired work
    % with Hugo Abreu.
    We first propose a quantum algorithm for the $\varepsilon$-spectral
    approximation of arbitrary tall and thin matrices of dimensions ${n \times
    d}$ with $n \gg d$ where each row has at most $S$ nonzero entries in
    $\smash{\tilde{O} \big(S\frac{\sqrt{nd}}{\varepsilon} + d)\big)}$ quantum
    queries and time $\smash{\tilde O(S\frac{\sqrt{nd}}{\varepsilon} +
    d^\omega)}$, which generalizes results by \citeauthor{apers_quantum_2020}
    \cite{apers_quantum_2020} that consider undirected weighted graphs. It is
    based on an approach by \citeauthor{cohen_uniform_2014}
    (\cite{cohen_uniform_2014}), where the time complexity is slightly improved
    by considering \emph{e.g.} Johnson-Lindenstrauss transforms. In the second part,
    we expose a practical application in convex optimization : approximate the
    central path of Interior-Point Methods.
\end{abstract}
\tableofcontents

\thispagestyle{firststyle}

\section{Introduction}\label{introduction}

Before the advent of quantum mechanics in the beginning of the \nth{20} century, physicists held three fundamental beliefs about the nature of the universe: 
\emph{determinism} (or \emph{causal determinism}), i.e.\ 
events in a given paradigm are bound by causality in such a way that any state is completely determined by prior states; 
\emph{reality}, i.e.\ the universe exists independently of observation; 
and \emph{locality} (or \emph{local causality}), i.e.\ what happens in a given space-time region should not influence what happens in another, space-like separated region. Quantum mechanics challenged each one of these notions \cite{nielsen_quantum_2010}.

Non-locality in particular posed the most problems: it was the corner-stone of Einstein, Podolsky and Rosen's argument in 1935 for the incompleteness of quantum mechanics \cite{epr}. 
Since it seemed to be possible for two spatially-separated parties sharing an entangled state to influence each other's measurements, resulting in \emph{faster-than-light communications}, they argued that such a behaviour was paradoxical (this became known as the \emph{EPR paradox}), and postulated that quantum mechanics should be compatible with a \emph{hidden local variable description}, in which non-local effects would not be allowed.

For more than 30 years, \emph{local-realist} description of quantum 
theory were not fully discarded. 
One had to wait until 1964, when \citeauthor{bell_einstein_1964}
%noticed that certain quantum correlations could not be accounted for by any theory which attributes only locally defined states to its basic physical objects\cite{bell_einstein_1964}.
proposed an experiment which would definitively decide whether or not certain physical effects of quantum entanglement could be reproduced
by local hidden variables: the \emph{Bell test} \cite{bell_einstein_1964}. He proposed a condition, known as \emph{Bell’s inequality}, that any physical experiment
has to satisfy if nature could be faithfully described by a classical local hidden variable theory. Subsequently, several experiments
% \hft{lesquelles ?} 
\cite{aspect_experiment} have been designed that have violated Bell’s inequality, proving 
that no hidden variable theory can describe certain phenomenons predicted by quantum mechanics.

Since the 1960s, the field of \emph{Bell nonlocality} has grown quite considerably \cite{brunner}, especially with the establishment of \emph{quantum information science}, where nonlocality plays a central role \cite{nielsen_quantum_2010}. It is at the heart of \emph{device-independent} (\emph{d.i.}) protocols such as \emph{self-testing} \cite{2020-self-testing-a-review}, \emph{d.i.\ quantum key distribution} \cite{di-qkd}, \emph{d.i.\ randomness generation} \cite{di-random}, and \emph{delegated blind quantum computing} \cite{blind-aaronson}.

%According to them, the formulation of quantum mechanics seemed to imply faster than light communication. 

% and it wasn't until the work of Bell in 1964 that it started being widely accepted as a fundamental consequence of the theory.

% Quantum mechanics challenged each one of these notions: in 1927, W. Heisenberg introduced the uncertainty principle which gives a fundamental limit to the accuracy certain pairs of physical quantities can be determined to from initial conditions; the Copenhagen interpretation 

% reality was put into question as well, since a particle can exist in superposition and only by observing it can you determine 

% Quantum mechanics challenged each one of these notions: [Heisenberg uncertainty principle => a precise knowledge at the quantum level is impossible] [wave-like properties of matter: things can exist in a supperposition of different states. (copenhagen interpretation of quantum mechanics)] [non-locality: space-like seperated states can exhibit correlations]

% Non-locality especially posed several problems: it was the corner-stone of Einstein, Podolsky and Rosen's argument in 1935 for the incompleteness of quantum mechanics:.


% locality = what happens in a given space-time region should not influence what 
% happens in another, space-like separated regiont
% determinism = 
% reality = the universe exists independent of observation

% [qm: paradox (EPR) => qm incomplete]
% [observations predited by quantum mechanics forced physicists to abandon these notions => neither locality neither realism hold]


% Quantum mechanics challenged each one of these notions: non-locality especially posed
% several problems, and was the corner-stone of Einstein's argument in 1935 for the 
% incompleteness of quantum mechanics. They argued that quantum theory should allow a 
% description with local hidden variables.

% [Bell 1964: qm => nonlocality]
% In 1964, Bell showed that quantum theory is incompatible with local hidden variables. 
% Since then, nonlocality has been studied extensively - notably in quantum information. 

% [self-testing]
% Several applications exist: notably self-testing and device-independent scenarios use
% expressions of nonlocality extensively.

% \hpa{put paragraph on applications: device-independent quantum key, randomness generation, self-test...}

\Autoref{seq:theoretical-framework} sets the stage for the rest of the report, establishing the relevant theoretical background and framework. Subsequently, deeper attention is given to the two main bodies of work: nonlocality detection (\Autoref{sec:nonloc-detection}) and the retrieval of Bell inequalities (\Autoref{seq:ineq-dual}). In \Autoref{sec:noise}, the limits of locality detection in noisy systems are studied, and lastly, in \Autoref{sec:non-linear}, a nonlinear approach to study of nonlocality is described.

\section{Leverage scores}
In order to estimate the importance of a row of a matrix relative to the others,
we define its leverage score.
\begin{definition}[Leverage score]\label{def:leverage-score}
Given a matrix $\A \in \mathbb R^{n \times d}$, the leverage score of $a_i^T$,
the $i$-th row of $\A$, is defined as
$$
\tau_i := a_i^T (\bs {A^TA})^+ a_i
$$
where $\A^+$ denotes the Moore-Penrose pseudoinverse of $\A$.
\end{definition}

In the algorithm, approximate leverage scores are computed, \textit{i.e.,\ }leverage scores of $\A$ according to some approximation $\B$, denoted \textit{generalized leverage scores}.

\begin{definition}[Generalized leverage scores]
Let $\A \in \mathbb R ^{n \times d}$ and $\B \in \mathbb R^{m \times d}$. The
leverage score of the $i$-th rows of $\A$ according to $\B$ is
$$
\tau_i^{\B}(\A) =
    \begin{cases}
        a_i^T(\bs{B ^T B})^+a_i & \text{if } a_i \perp \ker \B \\
        \infty & \text{otherwise}
    \end{cases} \, .
$$
\end{definition}

The case distinction comes from the fact that if $a_i \not\perp \ker \B$, the
generalized leverage score would be null, while $a_i$ could be the only row
pointing in its direction, and thus should be kept.

For the sake of conciseness, if $\B = \A$, we denote $\tau_i^{\B}(\A)$ by
$\tau_i(\A)$, and as long as the knowledge $\A$ is not important, we simply
denote $\tau_i(\A)$ by $\tau_i$.

If $\A$ and $\B$ are relatively close, \textit{i.e.\ }$\B$ is an
$\lambda$-spectral approximation of $\A$, it holds that
$$
\frac{1}{\lambda} \A^T\A \preccurlyeq \B^T\B \preccurlyeq \A^T\A \, ,
$$
which we write
$$
\A \approx_\lambda \B \, .
$$
% This yields the following proposition
\begin{theoremEnd}{theorem}[Leverage score approximate]
    Let $\A \in \mathbb R ^{n \times d}$ and $\B \in \mathbb R^{m \times d}$. If
    $\A \approx_\lambda \B$, then
    $$
        \tau_i(\A) \leq \tau_i^{\B}(\A) \leq \lambda \cdot \tau_i(\A) \, .
    $$
\end{theoremEnd}
\begin{proofEnd}
    By definition of $\A \approx_\lambda \B$, we have
$$
    \frac{1}{\lambda} \A^T\A \preccurlyeq \B^T\B \preccurlyeq \A^T\A \, ,
$$
    thus
$$
    (\A^T\A)^+ \preccurlyeq (\B^T\B)^+ \preccurlyeq \lambda \cdot (\A^T\A)^+ \, ,
$$
    which yields, using the definition of the generalized leverage scores and
    by the positive semi-definiteness of each of the matrices,
$$
    \tau_i(\A) \leq \tau_i^{\B}(\A) \leq \lambda \cdot \tau_i(\A) \, .
$$

\end{proofEnd}


\section{Grover search}
Considering the following function
\begin{equation*}
    f : [n] \> \{0, 1\} \quad \text{s.t.} \quad \big|\{i : f(i) = 1\}\big| = k \; ,
\end{equation*}
$f$ is the indicator of the rows kept in the matrix. In order to construct the
search oracle, we need to consider a list $\bs z$ whose $z_i$ are uniform random
number in $[0, 1]$ for all $i \in [n]$\footnote{$[n]$ shortens notation of the
set $ \{0, \cdots, n-1\}$}. $f$ is so that
\begin{equation} \label{eq:grover-oracle}
   f(i) = \begin{cases}
       1 & \text{if } z_i \leq p_i \\
       0 & \text{otherwise }
   \end{cases}\, .
\end{equation}
Note that there are two possibilities to find the $k$ marked elements.

\subsection{Repeated search}
With the naive approach of repeating $k$ Grover searches, it is very likely to
get less than $k$ disctinct marked elements. To address this issue, one can
repeat $\Theta(k \log k)$ times Grover search over the set $[n]$. This finds all
the distinct $k$ marked elements with a sufficiently high probability, see
\textit{e.g.} the coupon collector's problem. This requires $O(\sqrt{nk}\log k)$
= $\tilde O (\sqrt{nk})$ queries to $f$.

It is however possible to reduce the number of queries to $O(\sqrt{nk})$.

\subsection{Updating the list}
The list $\bs z$ of \textit{choices} is classically stored offline, and both the
access to an element and an update require time $O(1)$. Thus, after a call to
$f$, it is possible to efficiently update $\bs z$ in order to \textit{unmark}
the returned element. The following \autoref{alg:query-and-update} exposes how
the list $\bs z$ is updated at each iteration.

\begin{algorithm}[H]
    \caption{\textsc{QueryAndUpdate}($\bs z = \{z_i\}, f$)}
    \label{alg:query-and-update}
    \begin{algorithmic}[1]
        \State $i \<$ \textit{Grover search with $f$ over $[n]$}
        \State $z_i \< \infty$
        \State \Return $i$
    \end{algorithmic}
\end{algorithm}

Step 2 ensures that the $i$-th element is not marked anymore. At the $j$-th call
to the above procedure there remains $(k - j)$ marked elements, thus finding one
of them requires time $O(\sqrt{\frac{n}{k-j}})$. A single call to
\textsc{QueryAndUpdate} is thus as expensive as a call to $f$.


\begin{theoremEnd}{proposition}\label{prop:quantum-queries-grover}
    $k$ calls to \textsc{QueryAndUpdate} find $k$ distinct marked elements
    among $n$ in $O(\sqrt{nk})$ quantum queries.
\end{theoremEnd}\label{prop:k-call-qandu}
\begin{proofEnd}

Let $S$ be the total number of queries.

It can be expressed as a sum of all expected number of queries (we omit the big
$O$ notation on purpose):
\begin{equation*}
  S = \sum_{i = 0}^{k-1} \sqrt{\frac{n}{k-i}}
    = \sum_{i = 1}^{k} \sqrt{\frac{n}{i}}
    = \sqrt n \cdot \sum_{i = 1}^{k} i^{-\frac{1}{2}} \, .
\end{equation*}
Let $g(x) = x^{-\frac{1}{2}}$, since $g'(x) < 0 $ for all $x > 0$, $g$ is a
strictly decreasing function over $\mathbb R^+_*$. In addition, a formula to
bound a sum of a stricly decreasing function by integrals states that
\begin{equation*}
    \int_{a}^{b+1} g(s)ds \leq \sum_{i = a}^{b} g(i) \leq \int_{a-1}^{b} g(s)ds \, ,
\end{equation*}
which yields as long as $a = 1$ and $b = k$

\begin{table}[H]
    \setlength\tabcolsep{1.5pt}
    \centering
    \begin{tabular}{lccccc}
                      & $ \displaystyle\int_{1}^{k+1} g(s)ds $ & $\leq$ & $\displaystyle \sum_{i = 1}^{k} g(i)$ & $\leq$ & $\displaystyle \int_{0}^{k} g(s)ds$ \\
    $\Leftrightarrow$ & $\displaystyle \sqrt n \left[ 2 \sqrt s \right]_1^{k+1}$ & $\leq$ & $\sqrt n \displaystyle\sum_{i = 1}^{k} i^{-\frac{1}{2}}$ & $\leq$ & $\sqrt n \displaystyle \left[ 2 \sqrt s \right]_0^{k}$ \\
    $\Leftrightarrow$ & $\displaystyle 2 \sqrt n (\sqrt{k+1} - 1)$ & $\leq$ & $S$ & $\leq$ & $2 \sqrt{nk} \, .$
    \end{tabular}
\end{table}

Thus, $k$ calls to \textsc{QueryAndUpdate} find $k$ distinct marked
elements in $O(\sqrt{nk})$ quantum queries.

Note that herin we do not consider the time to construct $f$ at each iteration.
It is well when considering only the query complexity. Such implementation
requires a circuit of $O(\log n)$ qubits, and the associated quantum circuit
runs in time $O(h(n))$ as long as classicaly $f$ runs in time $O(h(n))$.
\end{proofEnd}


\section{Sampling procedure}

The sampling procedure is given query access to a matrix $\bs{A} \in
\mathbb{R}^{n \times d}$, a matrix $\bs{B} \in \mathbb{R}^{\tilde O(d) \times
d}$ and a number of rows to be sampled $k$. It then outputs a constant-factor
approximation $\bs{\hat{A}}$ of $\bs{A}$.

This sampling procedure, by using Grover's algorithm in the quantum setting,
avoids the explicit construction of the vector of leverage scores.

\begin{algorithm}[H]
  \caption{\textsc{Sample}($\bs{A}$, $\bs{B}$, $k$)}
  \label{alg:sample}
  \begin{algorithmic}[1]
    \Ensure $\bs{z} = \{z_i \sim \mathcal U (0, 1)\}_{i \in [n]}$
    \Comment{Stored offline.}

    \Ensure $\bs{A} \in \mathbb R ^{n \times d}$, $\bs{B} \in \mathbb R^{\tilde O(d) \times d}$ such that $A \approx_\lambda B$
    \State \textit{allocate $\bs{\hat{A}}$}
    \State \textit{define a function $f$ such that}
    \begin{equation*}
      f(i) = \begin{cases}
               1 & \text{if } z_i \leq \min \{1, \alpha \cdot a_i^{T}(\bs{B}^{T}\bs{B})^{+} a_i \cdot c \log d\} \text{,}\\
               0 & \text{otherwise.}
             \end{cases}
    \end{equation*}

    \Repeat
      \State $i \<$ \textsc{QueryAndUpdate}($\bs z$,$f$)
      \State \textit{add row $a_i$ to $\bs {\hat A}$, rescaled by $p_i^{-1/2}$}
    \Until{\textit{sampled $O(k)$ rows}}
    \State \Return $\bs {\hat A}$
  \end{algorithmic}
\end{algorithm}

One thing to note is that the parameter $k$ is normally set to $O(d \log d)$ (as
will be seen in more detail in \autoref{sec:uniform_sampling}), except when
computing an $\varepsilon$-spectral approximation, in which case $k$ is set to
$O(\frac{d\log d}{\varepsilon})$.


\section{Uniform sampling}
\label{sec:uniform_sampling}

\begin{figure*}
    \centering
    \begin{tikzpicture}
    % Matrices and sampling arrows
    \node [rectangle, line width = 0.25mm, draw, minimum width = 1cm, minimum height = .5cm, rounded corners = 0.1cm]
        (r) at (0,0) {$\A$};

    \draw [-stealth]
        (.8, 0) -- (1.2, 0) node[midway,above] {\footnotesize $1/2$};

    \node [rectangle, line width = 0.25mm, draw, minimum width = 1cm, minimum height = .5cm, rounded corners = 0.1cm]
        (r) at (2,0) {$\A_1'$};

    \draw [-stealth]
        (2.8, 0) -- (3.2, 0) node[midway,above] {\footnotesize $1/2$};
    \node []
        (r) at (4, 0) {$ \cdots $};
    \draw [-stealth]
        (4.8, 0) -- (5.2, 0) node[midway,above] {\footnotesize $1/2$};
    \node [rectangle, line width = 0.25mm, draw, minimum width = 1cm, minimum height = .5cm, rounded corners = 0.1cm]
        (r) at (6,0) {$\A_{L-1}'$};

    \draw [-stealth]
        (6.8, 0) -- (7.2, 0) node[midway,above] {\footnotesize $1/2$};
    \node [rectangle, line width = 0.25mm, draw, minimum width = 1cm, minimum height = .5cm, rounded corners = 0.1cm]
        (r) at (8,0) {$\A_L'$};

    \node [rectangle, line width = 0.25mm, draw, minimum width = 1cm, minimum height = .5cm, rounded corners = 0.1cm]
        (r) at (0,-1.5) {$\B$};

    \node [rectangle, line width = 0.25mm, draw, minimum width = 1cm, minimum height = .5cm, rounded corners = 0.1cm]
        (r) at (2,-1.5) {$\B_1'$};

    \node []
        (r) at (4, -1.5) {$ \cdots $};
    \node [rectangle, line width = 0.25mm, draw, minimum width = 1cm, minimum height = .5cm, rounded corners = 0.1cm]
        (r) at (6,-1.5) {$\B_{L-1'}$};

    \node [rectangle, line width = 0.25mm, draw, minimum width = 1cm, minimum height = .5cm, rounded corners = 0.1cm]
        (r) at (8,-1.5) {$\B_L'$};

    \node [rotate=90]
        (r) at (8, -.75) {$ = $};



    \draw [-stealth, rounded corners = 0.1cm, dashed]
        (1.4, -1.1) -- (0, -.4) -- (0, -1.1) node[midway, left, align=left] {\footnotesize $O(d\log d)$};
    \node [rotate=-25]
        (r) at (1, -.65) {\scriptsize $ \boldsymbol {\tau}^{\B_1'}(\A) $};

    \draw [-stealth, rounded corners = 0.1cm, dashed]
        (3.4, -1.1) -- (2, -.4) -- (2, -1.1) node[midway, left, align=left] {};
    \node [rotate=-25]
        (r) at (3, -.65) {\scriptsize $ \boldsymbol {\tau}^{\B_2'}(\A_1') $};

    \draw [-stealth, rounded corners = 0.1cm, dashed]
        (7.4, -1.1) -- (6, -.4) -- (6, -1.1) node[midway, left, align=left] {};
    \node [rotate=-25]
        (r) at (7, -.65) {\scriptsize $ \boldsymbol {\tau}^{\B_L'}(\A_{L-1}') $};
\end{tikzpicture}
    \caption{Structure of the recursive calls to \textsc{Approximate}. The
    leverage score vector $\bs \tau$ is indeed not explicitely computed, but is
    used for Grover search. The resulting matrix is $\bs B \approx_2 \bs A$.}
    \label{fig:sampling}
\end{figure*}

Uniform sampling does not uses $(R,k)$-reductions \textit{i.e.,}
Johnson-Lindenstrauss random projections as depicted in
\cite{li_iterative_2013}, and thus preserves the input matrix sparsity pattern,
which is usefull especially in the graph case. This will allow us, in this case,
to compute the leverage score of a single row in $O(1)$ quantum queries.

In our setting, computing leverages scores according to an approximation
$\bs{\tilde A}$ of the input matrix $\A$ requires to sample $$c \log d \sum_i
\tilde \tau_i$$ rows for a fixed constant $c$, where the $\tilde \tau_i$ are
computed according to $\bs{\tilde A}$.

In order to prove correctness of the algorithm we expose, it is convenient to
repesent the sampling of a matrix through the product with a diagonal matrix,
that we define below.

\begin{definition}
    Let  \emph{\texttt{Sample($\bs \tau, \alpha$)}} be a procedure that returns a
    diagonal matrix $S$ with independently chosen entries, and such $S_{ii} =
    p_i^{-1/2}$ with probability $p_i$ and $0$ otherwise, where $p_i = \min\{1,
    \alpha c \tau_i \log d\}$ for a fixed constant $c$.
\end{definition}

In the quantum setting, this is implemented by the procedure \textsc{Sample}
shown in \autoref{alg:sample}, thus there is no need to explicitly construct the
output matrix of the procedure. Nevertheless, it is necessary to show that the
two representations are equivalent.

\begin{theoremEnd}{claim}
    Given a matrix $\A$, on one hand let $\bs{S} = $ \emph{\texttt{Sample}}$(\bs 1,
    O(m/n))$ and consider $\bs{SA}$; on the other hand, let $\bs{A'}$ be a
    uniform ramdom subset of $O(m)$ rows of $\A$. Sampling according to
    $\bs{\tau^{SA}}$ is equivalent to sampling according to $\bs{\tau^{A'}}$.
\end{theoremEnd}

\begin{proofEnd}
A convient definition to consider is the one of the pseudoinverse of a vector.
\begin{definition}[Pseudoinverse of a vector]\label{def:vector-pseudoinverse}
   Let $x$ be a vector, the pseudoinverse of $x$ is
   $$
   x^+ = \begin{cases}
    \bs 0^T & \text{, if } x = \bs 0 \, ;\\
    \frac{x^*}{x^*x} & \text{, otherwise.}
   \end{cases}
   $$
\end{definition}

For the uniform sampling step, we consider $\bs S$ a sampling matrix with
$s_{ii} = 1$ \textit{w.p.} $\frac{m}{n}$ and $0$ otherwise. If $s_{ii} = 0$,
then $(sa_{i})^T$, the $i$-th row of $\bs{SA}$ is $\bs 0^T$, thus
$(sa_{i})^{T+}$ is $\bs 0$ : we can simply \emph{remove} this row since it does
not infere in the calculation of $\big((\bs{SA})^T\bs{SA}\big)^+$ according to
\autoref{def:vector-pseudoinverse}. Removing all the rows of $\A$ where $s_{ii}
= 0$ yields exactly $\bs{A'}$.
\end{proofEnd}
As long as the leverage score approximates used to sample are \textit{upper
bounds} on the true leverage score, \textit{i.e.,} for all $i$, $\tilde\tau_i
\geq \tau_i$, sampling $\A$ according to them yields a constant-factor spectral
approximation \cite{li_iterative_2013}.

\begin{theorem}[\cite{cohen_uniform_2014}]
    \label{thm:sum-expectation}
    Let $\A \in \mathbb R^{n\times d}$. Sampling uniformly at random $O(m)$ rows
    from $\A$ to form $\bs{\tilde A}$, implies, supposing one computes $\{\tilde \tau
    _i\}$ thanks to $\bs{\tilde A}$, that
    $$
    \mathbb E \left [ \sum_i \tilde \tau_i\right ] \leq \frac{nd}{m} \, .
    $$
\end{theorem}

The above theorem enables us to conclude on the number of rows to be sampled
given $m$.

\begin{theorem} [\cite{cohen_uniform_2014}]
    \label{thm:epsilon_spectral_approximation}
    Let $\A \in \mathbb R^{n\times d}$ suppose we sample uniformly $O(m)$ rows
    to form $\bs{A'}$. Computing $\tilde \tau_i = \min \{1,
    \tau_i^{\bs{A'}}(\A)\}$ and sampling $\A$ accordingly returns with high
    probability a constant factor spectral approximation of $\A$ with at most $O(\frac{nd\log
    d}{m})$ rows.
\end{theorem}

% and $\bs S = $ \emph{\texttt{Sample($\bs 1, \alpha$)}}.


Thus \autoref{thm:epsilon_spectral_approximation} implies correctness of
\autoref{alg:repeated-halving}.


\begin{algorithm}[H]
    \caption{\textsc{Approximate}($\bs A$)}
    \label{alg:repeated-halving}
    \begin{algorithmic}[1]
        \Ensure $\bs A \in \mathbb R ^{n \times d}$
        \State \textit{$\bs{A'} \<$ uniformly sample $\frac{n}{2}$ rows of $\bs A$}
        \If{$\bs{A'}$ \textit{ has more than $O(d\log d)$ rows }}
            \State $\bs{B'} \<$ \textsc{Approximate}($\bs{A'}$)
        \Else
            \State $\bs{B'} \< \bs {A'}$
        \EndIf
        % \State \textit{Set $\bs \tau = \Big\{ \tilde \tau_i = \min\{1, \tau_i^{\bs{A'}}(A)\} \Big\}_{i=1}^n$ }
        % \State \textit{Sample $\bs A$ according to $\bs \tau$ to form $\tilde A$}
        \State $\bs{B} \< $\textsc{Sample}$(\bs A, \bs B', O(d \log d))$
        \State \Return $\bs{B}$
    \end{algorithmic}
\end{algorithm}
Indeed, sampling $O(d \log d)$ rows is enough to obtain a constant factor
approximation: letting $m = O(n/2)$ in \autoref{thm:sum-expectation} yields
\begin{equation*}
\mathbbm E \left[ {\sum_i \tilde \tau_i} \right] = O(d) \, ,
\end{equation*}
hence
\begin{equation*}
    \log d \sum_i \tilde \tau_i = O(d \log d) \, .
\end{equation*}
The resulting matrix has $O(d\log d)$ rows and there are $O\big(\log
(\frac{n}{d\log d})\big)$ recursive calls to \textsc{Approximate}. It is
important to note that $\B$ is a
constant-factor-approximation of the matrix $\bs A$, which we write
$$\bs A \approx_{O(1)} \textsc{Approximate}(\bs A) \, .$$

\autoref{fig:sampling} shows graphically how the recursive sampling works.

It is possible to further speed up the calculation of leverage score by using
Johnson-Lindenstrauss transfom on each $\B_l'$, for all $1 \leq l \leq L$.

\subsection{Johnson-Lindenstrauss tranform}

For the sake of completeness, the Johnson-Lindenstrauss lemma is recalled in
\autoref{ap:jllemma}.

\subsubsection{Leverage score as a squared norm}

Let $\A \in \mathbb R^{n \times d}$, and recall that, for $a_i^T$ the $i$-th row
of $\A$, its leverage score $\tau_i$ is defined in \autoref{def:leverage-score}
as
$$
    \tau_i = a_i^T(\A^T\A)^+a_i \, .
$$
In order to express $\tau_i$ as a squared norm, let us denote
\begin{equation}\label{eq:def-xi}
    \bs \x_i := \A(\A^T\A)^+ a_i \, ,
\end{equation}
which yields the following proposition:
\begin{theoremEnd}{proposition}\label{prop:lv-as-norm}
    Given $\A$ an input matrix, one can compute the leverage scores of $\A$'s rows
    as follows:
    $$
        \tau_i(\A) = \|\bs \x_i \|_2^2 \, .
    $$
\end{theoremEnd}

\begin{proofEnd}
   Considering \autoref{eq:def-xi} yields
   \begin{equation*}
    \begin{aligned}
        \|\bs \x_i \|_2^2
            & = a_i^T \big((\A^T\A)^+\big)^T \A^T\A (\A^T\A)^+ a_i \\
            & = a_i^T \big((\A^T\A)^T\big)^+ \A^T\A (\A^T\A)^+ a_i \\
            & = a_i^T (\A^T\A)^+ \A^T\A (\A^T\A)^+ a_i \\
            & = a_i^T (\A^T\A)^+ a_i \\
            & = \tau_i(\A)
    \end{aligned}
   \end{equation*}
The second step comes from the commutation of the pseudoinversion with
transposition\footnote{$(\A^+)^T = (\A^T)^+$}, the third step follows
from the symmetry of $\A^T\A$, and the fourth step is the application of the
weak inverse property of pseudoinverses\footnote{$\A^+\A\A^+ = \A^+$}.
\end{proofEnd}


\subsubsection{Application of the JL transfom}

For our purpose, consider a matrix $\A \in \mathbb{R}^{\tilde O(d)\times d}$ and
let $\bs \x_i$ as introduced in \autoref{eq:def-xi}; thus, by
\autoref{prop:lv-as-norm}, $\|\bs \x_i\|_2^2 $ is exactly $ \tau_i$. Also, since
we can write $\Pi \bs \x_i = \Pi \A(\A^T\A)^+a_i$ with $\Pi \A(\A^T\A)^+ \in
\mathbb R^{\tilde O (\varepsilon^{-2})\times d}$, we denote $\| \Pi \bs \x_i
\|_2^2$ by $\hat \tau_i$. Thereby, we can restate \autoref{jl-lemma} as follows:

\begin{lemma}[DJL lemma, restated]
    For any $0 < \varepsilon < 1$, $\delta < 1/2$ and $d \in \mathbb N$, there
    exists a distribution over $\displaystyle \mathbb {R} ^{k\times d}$ from
    which the matrix $\Pi$ is drawn such that for $k =
    O\big(\varepsilon^{-2}\log(\delta^{-1})\big)$ and any vector $\x \in \mathbb
    R^d$, the following claim holds:
    \begin{equation*}
        \mathbb P \Big(\left | \hat \tau_i - \tau_i \right | > \varepsilon
        \cdot \tau_i \Big)<\delta \, .
    \end{equation*}
\end{lemma}

Let us denote by $X_i$ the event \guillemotleft $\left| \hat \tau_i - \tau_i
\right| > \varepsilon \cdot \tau_i$\guillemotright. Thus, taking the union
bound over all possible leverage scores, we have
\begin{equation*}
    \mathbb P (\bigcup_{i=1}^n X_i) \leq \sum_{i=1}^n \mathbb P(X_i) <
    \sum_{i=1}^n \delta = n \delta
\end{equation*}
Hence, setting $\delta = n^{-2}$ yields that, with probability $\geq 1-
\frac{1}{n}$, all leverage scores in the reduced space are an
$\varepsilon$-approximation of the original ones, \textit{i.e.,} for all $i \in [n]$,
\begin{equation}\label{eq:prob}
\left | \hat \tau_i - \tau_i \right | \leq  \varepsilon \cdot \tau_i \, .
\end{equation}


\begin{theoremEnd}{proposition}\label{prop:time-complexity-jl-lv}
    It is possible to compute a single leverage score in time $\hat
    O(\varepsilon^{-2}S)$, and such leverage score can be used to sample the
    input matrix and obtain a spectral approximation.
\end{theoremEnd}

\begin{proofEnd}
First, in order to have a more suitable expression of \autoref{eq:prob}, it is
convenient to break down the absolute value and examine both cases.
\begin{itemize}
    \item Case  $\hat \tau_i - \tau_i \leq 0$, then \textit{w.h.p.}
    \begin{table}[H]
        \centering
        \begin{tabular}{lrll}
        & $\hat \tau_i - \tau_i$ & $\geq$ & $- \varepsilon \cdot \tau_i$ \\
        $\Leftrightarrow$ & $\hat \tau_i$ & $\geq$ & $- \varepsilon \cdot \tau_i + \tau_i$ \\
        $\Leftrightarrow$ & $\hat \tau_i$ & $\geq$ & $(1 - \varepsilon) \cdot \tau_i$
        \end{tabular}
    \end{table}
    \item Case  $\hat \tau_i - \tau_i \geq 0$, then \textit{w.h.p.}
    \begin{table}[H]
        \centering
        \begin{tabular}{lrll}
            & $\hat \tau_i - \tau_i$ & $\leq$ & $\varepsilon \cdot \tau_i$ \\
            $\Leftrightarrow$ & $\hat \tau_i$ & $\leq$ & $\varepsilon \cdot \tau_i + \tau_i$ \\
            $\Leftrightarrow$ & $\hat \tau_i$ & $\leq$ & $(1 + \varepsilon) \cdot \tau_i$
        \end{tabular}
    \end{table}
\end{itemize}

Thus, still with probability $\geq 1 - \frac{1}{n}$, \autoref{eq:prob} can be
rephrased as follows,
$$
(1 - \varepsilon) \cdot \tau_i
    \; \leq \; \hat \tau_i
    \; \leq \; (1 + \varepsilon) \cdot \tau_i \, , \quad \forall i \in [n] \, .
$$
However, in order to obtain at the end a $\varepsilon$-spectral approximation
when sampling according to $\hat \tau_i$, it must hold that these approximations
are \textit{upper bounds} on the original ones. Note that here, what we denoted
by the \textit{original scores} are actually approximate of the \textit{true
scores}, \textit{i.e.,} the $\tilde\tau_i$. Thus, it suffices to sample
according to $\frac{1}{1-\varepsilon} \hat\tau_i$, since for all $i$, it holds
that
$$
\tilde \tau_i
    \; \leq \; \frac{1}{1-\varepsilon}\hat \tau_i
    \; \leq \; (\frac{1 + \varepsilon}{1-\varepsilon}) \tilde\tau_i \, ,
$$
which, by the way, implies that
$$
\mathbb E \left [\sum_{i=1}^n \tilde\tau_i \right ]
    \; \leq \; \mathbb E \left [\sum_{i=1}^n \frac{1}{1-\varepsilon}\hat \tau_i \right ]
    \; \leq \; \mathbb E \left [\sum_{i=1}^n \frac{1 + \varepsilon}{1-\varepsilon} \cdot \tilde\tau_i \right ] \, ,
$$
and equivalently, since the expectation of the sum of the approximate leverage
scores is bounded thanks to \autoref{thm:sum-expectation}, it holds that
$$
\frac{nd}{m}
    \; \leq \; \mathbb E \left [\sum_{i=1}^n \frac{1}{1-\varepsilon}\hat \tau_i \right ]
    \; \leq \; (\frac{1 + \varepsilon}{1-\varepsilon}) \frac{nd}{m} \, .
$$

Hence, it is possible to apply Johnson-Lindenstrauss transfrom to each
$\boldsymbol{B}_{l}'$ to get a matrix of dimension $\tilde O(\varepsilon^{-2})
\times d$, and thus computing a single leverage score in time $\tilde
O(\varepsilon^{-2}S)$.

\end{proofEnd}

\subsubsection{Time complexity of the scheme}\label{sec:time-complexity-jl}
Given as input a matrix $\A \in \mathbb{R}^{\tilde O(d) \times d}$, there is a
procedure with running time $\tilde O(d\varepsilon^{-2})$ that returns the
matrix $\Pi \in \mathbb R^{\tilde O(\varepsilon^{-2})\times \tilde O(d)}$ $-$
this procedure simply consists in setting each entry of $\Pi$ to independant
Gaussian random variable \cite{dasgupta_elementary_2003}. It takes time $\tilde
O (d^\omega)$ to obtain $\A(\A^T\A)^+$ and additional $\tilde
O(d^{2}\varepsilon^{-2})$ to compute the matrix-matrix product $\Pi
\A(\A^T\A)^+$. Thus, computing a single leverage score in time $\tilde
O(\varepsilon^{-2}S)$ requires a preprocessing time of $\tilde O(d^\omega)$. It
suffices to set $0 < \varepsilon < 1$ to be constant, so that each leverage
score requires time $\tilde O(S)$ to be computed. Doing so it holds, for any $0
\leq l \leq L$, that $\B'_l$ is still $O(1)$-spectral approximation of $\A'_l$
since the number of sampled rows stays unchanged and the leverage score used
remains upper bounds on the actual approximations, \emph{i.e.}, upper bounds on
the true leverage scores.

\subsection{Overall query complexity}
Let $L$ be the number of iterations required to obtain a matrix of $O(d\log d)$
rows. Thus, it holds that
$$
\frac{n}{2^L} = O(d \log d) \, ,
$$
hence, there is a total of
$$L = O(\log (\frac{n}{d \log d})) = \tilde O(1)$$
iterations.

Given a matrix $\A \in \mathbb{R}^{n \times d}$, let $S\leq d$ such that each
row of $\A$ has at most $S$ nonzero entries. Each call to \textsc{Sample}
computes $\tilde O(d)$ scores, where each score requires $O(S)$ queries. Since
each matrix has fewer than $n$ rows, the total number of rows of all of the
matrices together can be bounded by $Ln = \tilde O(n)$. Thus, by
\autoref{prop:quantum-queries-grover}, sampling $\tilde O(d)$ rows among $O(n)$
thoughout the $L$ iterations makes a total of $\tilde O(S\sqrt{nd})$ quantum
queries.

Furthermore, it is possible to store implicitly each reduced matrix, by
implicitly keeping track of the discarded rows through a string as shown in
\autoref{sec:data-structure-matrix}. However the last matrix has to be
explicitly written in order to compute $(\bs{A_L'}^T \bs{A_L'})^+$ : this
requires additional $\tilde{O}(dS)$ queries. Hence, the query complexity of
\autoref{alg:repeated-halving} is $\tilde{O} (S(\sqrt{nd} + d))$.

% Thus, there is a total of $\tilde O(d \log d \varepsilon^{-2})$ rows to be
% queried from the matrix while halving, among $O(n)$ rows, which requires $\tilde
% O(\sqrt{\frac{nd}{\varepsilon^2}})$ quantum queries thoughout Grover's
% iteration.

\section{Time complexity analysis}
To obtain the time complexity of the whole procedure, it suffices considering
the time of a single iteration, and since there is a logarithmic number of
iterations, considering all of them fits in the $\tilde O$ notation.

% At first, it is clear that given $\bs B \in \mathbb R^{O(d\log d)\times d}$,
% computing $(\B^T\B)^+$ takes time $\tilde O(d^\omega)$, and this is done only
% once per iteration.



% After that, we need to obtain the time complexity of calculating a single
% leverage score. Using \autoref{prop:lv-as-norm}, and
% \autoref{prop:time-complexity-jl-lv} and considering $\Pi \B (\B^T\B)^+$ known,
% calculating an approximate leverage score takes time $\tilde O(S)$ per score and
% has to be done $\tilde O(\sqrt{nd})$ times per iteration. The knowledge of $\Pi
% \B (\B^T\B)^+$ requires additional time $\tilde O(d^\omega)$.

In order to end up with a sublinear time algorithm, it is necessary to avoid
representing explicitly the intermediate matrices $\A'_l$.


\subsection{Data structure for matrix sampling}\label{sec:data-structure-matrix}

In order to represent each subsampling $\A'_l$ of the original matrix, a random
$n$-bit-string $Z_l$ is associated to the $l$-th iteration, and is such that

\begin{equation}\label{eq:z-l-i}
    (Z_l)_i\footnote{Given a string $z$, $(z)_i$ denotes its $i$-th bit.} =
    \begin{cases}
        1 & \text{ w.p. } \frac 1 2 \, , \\
        0 & \text{otherwise.}
    \end{cases}
\end{equation}

% At a given iteration $l$, when querying one line $i$, one wants to query a
% single element of $Z'_l$, its $k$-bit, which we denote $(Z'_l)_k$

% \begin{equation}
%   \label{eq:element-product}
%   (z')_l^i = z_l^1 \wedge \cdots \wedge z_l^i \text{,}
% \end{equation}
% with $z_i^k$ corresponding to the $k$-th bit of $Z_i$.



\begin{theoremEnd}{claim}
    Since the set of the rows of $\A'_l$ must be a subset of those of $\A'_{l -
    1}$, the matrix $\A'_l$ can be represented through a bit-string, denoted $Z'_l$,
    obtained by doing the element-wise product (logical conjunction) of the
    \emph{previous}
    strings. That is,

    \begin{equation}\label{eq:zprime-l-i}
        (Z'_l)_i = \displaystyle\bigwedge_{k \leq l} (Z_k)_i, \quad \forall i \in [n] \, .
    \end{equation}
\end{theoremEnd}

\begin{proofEnd} By simple probabilistic argument. Let $1 \leq l \leq L$. For any
$1 \leq k \leq l$, it holds by \autoref{eq:z-l-i} that the $i$-th bit of  $Z_k$
is 1 with probability half for all $i$ in $[n]$, which implies that
$$
(Z'_l)_i = \wedge_{k=1}^l (Z_k)_i = 1 \textit{ w.p. } \frac{1}{2^l} \, .
$$

Thus, the expected number of ones in $Z'_l$ is $\frac{n}{2^l}$, which is exactly
the expected number of rows of $\A_l$.
\end{proofEnd}
It would not be possible to explicitly compute this string while staying in
sub-linear time. In any cases, at each iteration $\tilde O(\sqrt{nd}) < n$
queries are done to the matrix, so the full string doesn't need to be explicitly
constructed.


\subsection{$k$-wise independent strings}
We use $k$-wise independent string to simulate access to the random string. The
result we use is the following theorem from \cite{zhandry_secure_2015}.

\begin{theorem} \label{thm:q-query-alg}
    Any $q$-query algorithm cannot distinguish between a uniformly random string
    and a $2q$-wise independent string.
\end{theorem}

This allows us to discard an explicit string $Z_l$ of size $O(n)$ and use
instead a $k$-wise independent string with $k \in \tilde O(\sqrt{nd})$. This is
achived thanks to $k$-independent hashing functions whose definition is recalled
in \autoref{ap:def-k-independent}.

From \autoref{thm:k-indep-data-structure}, we can assert that such a data
structure can be constructed in time $\tilde{O}(\sqrt{nd})$. An access to a bit
is done in time $\tilde{O}(1)$ and thus obtaining a single bit of $Z'_l$ as
depicted in \autoref{eq:zprime-l-i} takes time $\tilde O (1)$ since $L = \tilde
O(1)$.

Hence, at first -- recalling the result of \autoref{sec:time-complexity-jl} --
we compute a single leverage score in time $O(S)$ with an additional
preprocessing time of $O(d^\omega)$. Thus the whole sampling procedure, which
yields an approximation $\B$ of the input matrix $\A$ the $\tilde O(d)$ rows and
such that
$$
\A \approx_{O(1)} B
$$
is achieved in time $\tilde O(S\sqrt{nd} + d^\omega)$.


\subsection{$\varepsilon$-spectral approximation}
The ouput matrix of \textsc{Approximate} is a $O(1)$-spectral approximation of
the input matrix $\A$ with $O(d\log d)$ rows. In order to obtain, in the end, an
$\varepsilon$-spectral approximation $\B$ of $\A$, \emph{i.e.,} a matrix $\B \in
\mathbb R^{O(\varepsilon^{-2}d\log d) \times d}$ such that, for $\varepsilon >
0$ and for all $\x \in \mathbb R^d$,
$$
(1-\varepsilon) \x^T\A\x\leq \x^T\B\x\leq (1+\varepsilon)\x^T\A\x \, ,
$$
it suffices, with input $\A$ and as long as we obtained $\A \approx_{O(1)} \B$,
to sample $O(\varepsilon^{-2} d \log d )$ rows of $\A$ according to $\bs
\tau^{\B}(\A)$. This additional step requires $\smash{\tilde
O(S\frac{\sqrt{nd}}{\varepsilon})}$ quantum queries and supplementary time of
$\smash{\tilde O(S\frac{\sqrt{nd}}{\varepsilon} + d^\omega)}$.

\section{Application}
There are several applications of matrix sparsification, and the one we chosed
to depict is convex optimization, and more precisely the interior points methods
(IPM). A quick summary on IPM -- with the explicit algorithm -- is provided in
\autoref{ap:ipm}.

We denote herin the distance between two vectors $u, v \in \mathbb R^{n}$ by the
following weighed norm with respect to the operator $Q$, which we denote the
$Q$-induced norm,
\begin{equation}\label{eq:def-h-norm}
\| u - v \|_Q = \sqrt{(u-v)^TQ(u-v)}\, ,
\end{equation}
to quantify the convergence rate. For our purpose, we denote by $x$ the actual
\emph{current} value of Newton's iterations, by $y$ the approximated one, and by
$x'$ and $y'$ the result of a Newton's iteration starting from $x$ and $y$
respectively; the start superscript denotes the minimizer of $\Phi$. We want to
bound $\|y' - x^* \|$, so that it describes a path that converges towards to the
actual minimizer.


\subsection{Approximate Hessian}

Considering a $\lambda$-spectral approximation of $S^{-1}B$, and taking into
account that our Hessian is $H = B^TS^{-2}B$, we denote the approximated Hessian
$\tilde H$ for some $\lambda \geq 1$. First, note that this implies
\begin{equation}\label{eq:hessian-bounds}
    \begin{aligned}
        & \frac{1}{\lambda} H \preccurlyeq \tilde H \preccurlyeq H \\
\Leftrightarrow \quad & H^{-1} \preccurlyeq \tilde H^{-1} \preccurlyeq \lambda H^{-1} \, .
    \end{aligned}
\end{equation}
At first, we consider an iteration  where the initial points are equal. In order
to use the above inequality, we need to express $\| x' - y' \|$ in terms of
both Hessians, \emph{i.e.,} considering Newton's step. That is,
\begin{equation} \label{eq:vector-distance}
    \begin{aligned}
        \left\| x' - y' \right\|
            & = \left\| \Big(x - H^{-1}\nabla\Phi_\mu(x)\Big) - \Big(y - \tilde H^{-1}\nabla\Phi_\mu(y)\Big)\right\| \\
            & = \left\| x - H^{-1}\nabla\Phi_\mu(x) - x + \tilde H^{-1}\nabla\Phi_\mu(x)\right\| \\
            & = \left\| \tilde H^{-1}\nabla\Phi_\mu(x) - H^{-1}\nabla\Phi_\mu(x) \right\| \\
            & = \left\| \Big[ \tilde H^{-1} - H^{-1}\Big ] \nabla\Phi_\mu(x) \right\| \, . \\
    \end{aligned}
\end{equation}
Thus, it suffices to bound $\big[ \tilde H^{-1} - H^{-1}\big ] $ to obtain
bounds on $\left\| x' - y' \right\|$. A straightforward consequence of
\autoref{eq:hessian-bounds}, is that
\begin{equation*}
\boldsymbol 0 \preccurlyeq \tilde H^{-1} - H^{-1} \preccurlyeq (\lambda - 1) H^{-1} \, ,
\end{equation*}
where $\boldsymbol 0$ is a full-zero matrix.
Therefore, it holds for all vectors, and especially $\nabla \Phi_\mu(x)$, that
\begin{equation*}
    0 \leq \left\| \Big[ \tilde H^{-1} - H^{-1}\Big ] \nabla\Phi_\mu(x) \right\|
 \leq \left\| (\lambda - 1) H^{-1} \nabla\Phi_\mu(x) \right\| \, ,
\end{equation*}

Note that $\left\| H^{-1} \nabla\Phi_\mu(x) \right\|$ is equal to $\|x' - x \|$
\emph{i.e.,} the length of the \emph{actual} Newton's step. As long as $\lambda
< 2$, $y'$ is closer to $x'$ than was $x$. The above equation can thus be
equivalently rewritten thanks to \autoref{eq:vector-distance} as
\begin{equation} \label{eq:x1-y1-bound}
    0 \leq \left\| x' - y' \right\|
 \leq (\lambda - 1) \left\|x' - x \right\| \, .
\end{equation}

In other words, it is possible to bound the distance bewteen $x'$, the actual
point we would like to obtain, and its approximation $y'$ within a
multiplicative factor as long as we consider $\lambda$-spectral approximation of
the Hessian.

In order to prove convergence of the Newton's method when we consider
$\lambda$-spectral approximation of the Hessian, we bound $\|y' - x^* \|$ in
terms of $\|x - x^*\|$. It holds that
\begin{equation}\label{eq:bound-opt-approx-hessian}
    \begin{aligned}
       \| y'-x^*\|
            & =\left\| y' - x' + x' - x^* \right\| \\
            & \leq \left\| y' - x' \| + \| x' - x^* \right\| \\
            & \leq (\lambda - 1) \left\| x - x' \right\| + \frac 1 2 \left\| x - x^*\right\|^2 \\
            & \leq (\lambda - 1) \left\| x - x^* \right\| + \frac{1}{2} \left\| x - x^* \right\| \\
            & = (\lambda - \frac{1}{2}) \left\| x - x^*\right\| & ,\\
    \end{aligned}
\end{equation}
which bounds the step of an approximated iteration. However, this enforces $ 1
\leq \lambda < \frac 3 2$ (\emph{i.e.,} choose $\lambda = 1 + \varepsilon$ for
an $\varepsilon \ll \frac 1 2$) in order to have $\lambda - \frac{1}{2} < 1$ and
hence a step \emph{towards} the minimizer $y^*$ ($y'$ in a \emph{ball} of radius
$ <\| x - x^*\|$ around $x^*$).

\subsection{Gradient approximation}
The goal here is to obtain an approximation of $\nabla \Phi$, the gradient
involved in the interior points method.

\subsubsection{Initial approach}\label{sec:initial-approach}

To do so, we'll consider a vector of random variables, such that its mean is
equal to our gradient, using result of \citeauthor{Cornelissen_2022} which
exposes a quantum algorithm for estimating the mean of a $n$-dimensional random
variable; they provide the following informal theorem.

\begin{theorem}[\cite{Cornelissen_2022}]\label{th:estimate-mean}
    Given a $n$-dimensional random variable $X$, there exists a quantum
    algorithm that returns with high probability an estimate $\tilde \mu$ of the
    mean $\mu$ of $X$ such that
\begin{equation*}
    \| \tilde \mu - \mu \|_2 \leq \frac{\sqrt{n \emph{tr}(\Sigma_X)}}{T}
\end{equation*}
in $\tilde O(T)$ queries as long as $T > n$.
\end{theorem}
Note that the bound is obtained in 2-norm, but the convergence rate of Newton's
method is in terms of the $H$-norm. As usual, we use $B \in \mathbb R^{m \times
n}, s \in \mathbb R^{m}$ and $H = B^T diag(s)^{-2} B$ where we denote $diag(s)$ by $S$. Let $X = B^T
S^{-1}\ket{e}$, where $e \sim \mathcal{U}[m]$. We denote $h := \sum_e B^T S^{-1}
\ket e = B^TS^{-1} \ket{\bs 1} $, where $\ket {\bs 1}$ is the full-one vector,
thus  $\mu_X := \esp X = \frac 1 m B^TS^{-1}\ket{\bs 1} = \frac 1 m h$. We
denote by $\Sigma_Z$ the covariance matrix of the $n$-dimensional random
variable $Z$. The goal is to obtain an $\tilde h$ such that $\|h - \tilde
h\|_{H^{-1}} \leq \varepsilon$ . In other words, we want $\tilde\mu_X$ such that
\begin{equation*}
    \|\mu_X - \tilde \mu_X \|_{H^{-1}} \leq \frac \varepsilon m
\end{equation*}
which, by using \autoref{prop:change-norm}, means
\begin{equation*}
    \left\|H^{-\frac 1 2} \mu_X - H^{-\frac 1 2}\tilde \mu_X \right\|_{2} \leq \frac \varepsilon m \, ;
\end{equation*}
that is,
\begin{equation} \label{eq:estimate-bound}
    \| \mu_Y -  \tilde \mu_Y \|_{2} \leq \frac \varepsilon m \, ,
\end{equation}
for a certain random variable $Y$.
As such, let $Y = H^{-\frac 1 2}X = H^{-\frac 1 2} B^T S^{-1}\ket{e}$ where $e
\sim \mathcal{U}[m]$, hence $\mu_Y = H^{-\frac 1 2} \mu_X$; we use the result of
\cite{Cornelissen_2022} to get the $\tilde \mu_Y$.


The covariance matrix of $Y$ is
\begin{equation*}
\begin{aligned}
\Sigma_Y &= \esp{YY^T} - \esp{Y}\esp{Y^T} \\ &= H^{-\frac 1 2} \Sigma_X H^{-\frac 1 2} - \esp{Y}\esp{Y^T}\, ,
\end{aligned}
\end{equation*}
where
\begin{equation*}
    \Sigma_X = \frac{1}{m} H - \frac{1}{m^2}h h^T
    \quad \text{and} \quad
    \esp{Y} = \frac{1}{m}H^{-\frac 1 2}h \, ,
\end{equation*}
hence
\begin{equation*}
    \Sigma_Y = \frac 1 m \mathbbm 1 - \frac{2}{m^2} H^{-\frac 1 2} h h^T H^{-\frac 1 2} \preccurlyeq \esp{YY^T} \, .
\end{equation*}

\begin{theoremEnd}{claim}\label{clm:sigma-y-bound}
    $\Sigma_Y \preccurlyeq \frac 1 m \mathbbm 1$ .
\end{theoremEnd}
\begin{proofEnd}
In order to prove \autoref{clm:sigma-y-bound}, we'll use the properties of
positive-(semi)definite matrices shown in \autoref{sec:psd-prop} to show that
$\frac{1}{m^2} H^{-\frac 1 2} h h^T H^{-\frac 1 2} \succcurlyeq 0$, where $H = A
A^T, A = B^TS^{-1}$ and $h = B^TS^{-1}\ket{\bs 1}$.

Since $H = A A^T$, $H \succcurlyeq 0$\footnote{We can assume $H \succ 0$ since --
in the context of IPM -- $H$ is nonsingular.} thus by
\autoref{prop:psd-inverse} $H^{-1} \succ 0$ and by \autoref{prop:psd-root}
$H^{-\frac{1}{2}} \succ 0$ Hence,
\begin{equation*}
\begin{aligned}
 H^{-\frac{1}{2}} h h^T H^{-\frac{1}{2}}
    & = (H^{-\frac{1}{2}}h)(h^TH^{-\frac{1}{2}}) \\
    & = (H^{-\frac{1}{2}}h)(H^{-\frac{1}{2}}h)^T \\
    & \succcurlyeq 0
\end{aligned}
\end{equation*}
which implies,
    $$ \Sigma_Y \preccurlyeq \frac 1 m \mathbbm 1 - \frac{1}{m^2} H^{-\frac 1 2} h h^T H^{-\frac 1 2} \preccurlyeq \displaystyle\frac 1 m \mathbbm 1$$
\end{proofEnd}
Using \autoref{th:estimate-mean} with the random variable $Y$ yields $\tilde \mu_Y$ such
that
\begin{equation}\label{eq:apply-th-y}
\| \mu_Y - \tilde \mu_Y\|_2 \leq \frac{\sqrt{n \text{tr}(\Sigma_Y)}}{T} \, .
\end{equation}
Having $\Sigma_Y  \preccurlyeq \frac 1 m \mathbbm 1$ implies
$\text{tr}(\Sigma_Y) \leq \frac{n}{m}$ since $\Sigma_Y \in \mathbb R^{n \times
n}$. Following our initial assumption -- \autoref{eq:estimate-bound} -- we want
the left side of \autoref{eq:apply-th-y} to be smaller than $\frac \varepsilon
m$, \emph{i.e.},
\begin{equation*}
    \frac{n}{\sqrt m T} \leq \frac \varepsilon m \, ,
\end{equation*}
which is achieved if and only if
\begin{equation*}
    T \geq \frac{n \sqrt m}{\varepsilon} \, .
\end{equation*}


\subsubsection{Using sparsified Hessian}
We consider we have a $\lambda$-spectral approximation $\tilde{H}$  of $H$ such
that
\begin{equation}\label{eq:approx}
\frac{1}{\lambda} H \preccurlyeq \tilde H \preccurlyeq H \, ,
\end{equation}
which we can rephrase
\begin{equation} \label{eq:approx-root}
\frac{1}{\sqrt \lambda} H^{\frac{1}{2}} \preccurlyeq \tilde H^{\frac{1}{2}} \preccurlyeq  H^{\frac{1}{2}} \, ,
\end{equation}
and we'll use another straightforward formulation, \emph{i.e.},
\begin{equation} \label{eq:approx-inverse-root}
H^{-\frac{1}{2}} \preccurlyeq \tilde H^{- \frac{1}{2}} \preccurlyeq \sqrt \lambda H^{-\frac{1}{2}} \, .
\end{equation}
In a similar manner as in \autoref{sec:initial-approach}, we define $\tilde Y =
\tilde H^{-\frac{1}{2}}X = \tilde H^{-\frac{1}{2}} B^T S^{-1}\ket{e}$ where $e
\sim \mathcal{U}[m]$. This yields
\begin{equation*}
    \begin{aligned}
        \esp{\tilde Y \tilde Y^T}
            & = \tilde H^{-\frac 1 2} \esp{XX^T}  \tilde H^{-\frac 1 2} \\
            & = \frac 1 m \tilde H^{-\frac 1 2} H \tilde H^{-\frac 1 2} \, .
    \end{aligned}
\end{equation*}
From \autoref{prop:product-of-root-psd}, we deduce
\begin{equation*}
    \frac {1}{m} \mathbbm 1\preccurlyeq \esp{YY^T} \preccurlyeq \frac{\lambda}{m} \mathbbm 1.
\end{equation*}
Tracing each part, and using \autoref{clm:sigma-y-bound} yields
\begin{equation*}
    \text{tr}(\Sigma_{\tilde{Y}}) \leq \frac{n\lambda}{m} \, .
\end{equation*}
The \autoref{th:estimate-mean} ensures the existence of $\tilde
\mu_{\tilde Y}$ such that
\begin{equation*}
\| \mu_{\tilde Y} - \tilde \mu_{\tilde Y} \|_2
    \leq \frac{\sqrt{n \text{tr}(\Sigma_{\tilde Y})}}{T}
    \leq \frac{n \sqrt \lambda}{\sqrt m T}.
\end{equation*}
Recall that we want $\| \mu_{\tilde Y} - \tilde \mu_{\tilde Y} \|_2 \leq \frac
\varepsilon m$. As such, we set
\begin{equation*}
T \geq \frac{n \sqrt{m\lambda}}{\varepsilon} \, .
\end{equation*}
Therefore, choosing $T$ as defined above yields
\begin{equation*}
\| \mu_{\tilde Y} - \tilde \mu_{\tilde Y} \|_2
    = \left\| \tilde H^{- \frac 1 2} (\mu_{X} - \tilde \mu_{X}) \right\|_2
    = \left\| \mu_{X} - \tilde \mu_{X} \right\|_{\tilde H^{-1}} \leq \frac \varepsilon m,
\end{equation*}
and consequently
\begin{equation}\label{eq:h-tilde-bound}
    \|h - \tilde h \|_{\tilde H^{-1}} \leq \varepsilon \, ,
\end{equation}
which is a quantity preserved by the initial sparsification: \autoref{eq:approx}
ensures that for all $\x \in \mathbb R^n$, the following holds :
\begin{equation*}
\sqrt{\x^T H^{-1} \x} \leq \sqrt{\x^T \tilde H^{-1} \x } \leq \sqrt{\lambda} \cdot \sqrt{\x^T H^{-1} \x} \, ,
\end{equation*}
hence, by definition of $\| \cdot \|_{H^{-1}}$, we have
\begin{equation*}
    \|h - \tilde h \|_{H^{-1}}
    \leq \|h - \tilde h \|_{\tilde H^{-1}}
    \leq \sqrt \lambda \|h - \tilde h \|_{H^{-1}} \, .
\end{equation*}
Note that the above inequality would have hold if we were doing $T =
\frac{n\sqrt m}{\varepsilon}$ queries to compute the approximate with respect to
both $H^{-1}$ and $\tilde H^{-1}$. However, to compute $\tilde h$ and bound it
with respect to $\| \cdot \|_{\tilde H^{-1}}$, we do $\sqrt \lambda T$ queries,
and as such, \autoref{eq:h-tilde-bound} effectively holds.


\subsubsection{Application within Newton's step -- Measure of progress}
We consider we have an approximate of the gradient $\nabla \Phi$, $\tilde \nabla
\Phi$, such that for all $\x \in \mathbb R^n$, it holds that
\begin{equation}\label{eq:approx-gradient-bound}
    \left\|\nabla \Phi (\x) - \tilde \nabla \Phi (\x) \right\|_{H^{-1}} \leq \varepsilon \, ,
\end{equation}
where $H$ is the current (in terms of the Newton's steps) Hessian matrix. The
above inequality can be directly used to bound the convergence of Newton's
method. We first recall that
\begin{equation*}
   \begin{aligned}
    x' = x - H^{-1}\nabla\Phi(x) \, ; \\
    y' = x - \tilde H^{-1}\tilde \nabla \Phi(x)\, ,
   \end{aligned}
\end{equation*}
and we define
\begin{equation*}
    \hat y = x - H^{-1} \tilde \nabla \Phi(x) \, ,
\end{equation*}
where we only use the approximate gradient, the Hessian is the actual one. The
measure of progress can be expressed as follows, using the triangular inequality
\begin{equation*}
    \| y'- x^* \|_H \leq \| y'- \hat y \|_H + \| \hat y - x' \|_H + \| x' - x^* \|_H \, ,
\end{equation*}
where we'll bound each of the three terms. The first term the least
straightforward to bound.
\begin{equation}\label{eq:bound-first-term}
    \begin{aligned}
        \| y' - \hat y \|_H
            & = \|\tilde H^{-1}\tilde \nabla \Phi(x) - H^{-1}\tilde \nabla \Phi(x) \|_H \\
            & = \| H^{\frac{1}{2}}\tilde H^{-1}\tilde \nabla \Phi(x) - H^{\frac{1}{2}}H^{-1}\tilde \nabla \Phi(x) \|_2 \\
            & = \| H^{\frac{1}{2}}\tilde H^{-1}H^{\frac{1}{2}}(H^{-\frac{1}{2}}\tilde \nabla \Phi(x)) \\
            & \qquad - H^{\frac{1}{2}}H^{-1}H^{\frac{1}{2}}(H^{-\frac{1}{2}}\tilde \nabla \Phi(x)) \|_2 \\
            & = \|(H^{\frac{1}{2}}\tilde H^{-1}H^{\frac{1}{2}} - \mathbbm 1) (H^{-\frac{1}{2}}\tilde \nabla \Phi(x))\|_2 \\
            & \leq (\lambda - 1) \left(\|x - x^*\|_H - \varepsilon\right) \, .
    \end{aligned}
\end{equation}
% \hft{Explain the steps}

The first step is the straight application of \autoref{prop:change-norm}. The
second step is a multiplication by $H^{\frac 1 2}H^{-\frac 1 2} = \mathbbm 1$ in
order to factorize and simplify in step 3. The last step is obtained as follows:
we use triangular inequality to separate it into a product of norms, and on the
one hand we have
\begin{equation*}
    \begin{aligned}
        \| H^{-\frac{1}{2}}\tilde \nabla \Phi(x)\|_2
        & \leq \| H^{-\frac{1}{2}} \nabla \Phi(x)\|_2 + \varepsilon\\
        & = \| H^{-1} \nabla \Phi(x)\|_H + \varepsilon \\
        & = \| x' - x \|_H + \varepsilon \\
        & \leq \| x - x^*\|_H + \varepsilon
   \end{aligned}
\end{equation*}
assuming that Newton's method does not overshoot. And on the other, hand we use
\autoref{prop:product-of-root-psd} together with \autoref{eq:approx} to obtain
\begin{equation*}
    \|H^{\frac{1}{2}}\tilde H^{-1}H^{\frac{1}{2}} - \mathbbm 1 \|_H \leq \lambda - 1 \, .
\end{equation*}
For the second term, from \autoref{eq:approx-gradient-bound} we have
\begin{equation}\label{eq:bound-second-term}
   \begin{aligned}
    \| \hat y - x' \|_H
        & = \|H^{-1}\nabla\Phi(x) - H^{-1}\tilde\nabla\Phi(x) \|_H \\
        & = \|\nabla\Phi(x) - \tilde\nabla\Phi(x) \|_{H^{-1}} \\
        & \leq \varepsilon
   \end{aligned}
\end{equation}
The last one is bounded by the convergence rate of the exact Newton's method
(see \emph{e.g.,} \cite[p. 43]{numerical_2006}); such that
\begin{equation}\label{eq:bound-third-term}
    \| x' - x^* \|_H \leq \frac{1}{2} \| x - x^*\|^2_H \, .
\end{equation}
Gathering \autoref{eq:bound-first-term}, \autoref{eq:bound-second-term}, and
\autoref{eq:bound-third-term}, we obtain the following measure of progress.
\begin{equation*}
    \begin{aligned}
        \| y' - x^* \|_H
            & \leq (\lambda - 1) \left(\|x - x^*\|_H - \varepsilon\right) + \varepsilon + \frac{1}{2} \| x - x^*\|_H \\
            & \leq (\lambda - \frac{1}{2})\|x - x^*\|_H + (2 - \lambda) \varepsilon \, .
    \end{aligned}
\end{equation*}

In order to have, let's say, $\lambda - \frac{1}{2} = \frac{9}{10}$, we set
$\lambda = \frac{14}{10}$. Recall that $\lambda$ is the approximation factor for
the sparsification of the Hessian, and from
\autoref{eq:bound-opt-approx-hessian}, we wanted $1 \leq \lambda < \frac{3}{2}
$, which is maintained here. In order to end up with a \emph{clean} expression
for the measure of progress, we set $\varepsilon \leftarrow \frac{3}{5}
\varepsilon$, and as such, we have
\begin{equation*}
        \| y' - x^* \|_H \leq \frac{9}{10}\|x - x^*\|_H + \varepsilon \, .
\end{equation*}

With this measure of progress, we can prove that we converge towards the
minimizer $x^*$ when we consider the approximate procedure. More formally we
show that $y^*$ is in an $O(\varepsilon)$-ball around $x^*$.

\subsection{Convergence of the Approximate Newton's method}

See \autoref{sec:recursive-sequence} for the method for finding the
$k$-th term of a recursive sequence. We first recall that we have
the following bound
\begin{equation}\label{eq:upper-bound-a-1}
    \| y_1 - x^* \|_H \leq \frac{9}{10} \| y_0 - x^* \|_H + \varepsilon \, ,
\end{equation}
as such we can define the following recursive sequence :
\begin{equation}\label{eq:definition-f-ipm}
    \begin{cases}
        a_0 = \| y_0 - x^* \|_H \\
        a_{k+1} = \frac{9}{10} a_k + \varepsilon
    \end{cases} \, ,
\end{equation}
It is easy to see that \autoref{eq:upper-bound-a-1} shows an upper bound on
$a_1$. For the sake of simplicity, we can set $a_1 = \| y_1 - x^* \|_H$.
\begin{theoremEnd}{claim}
    With \autoref{clm:a-k+1-value}, we show that
    \begin{equation*}
        a_{k+1} = (\frac{9}{10})^{k+1} a_0 + O(\varepsilon) \, .
    \end{equation*}
    proving that our procedure converges.
\end{theoremEnd}
\begin{proofEnd} Here $p = \frac{9}{10}$ and $q = \varepsilon$.
    \begin{equation}\label{eq:valueof-ak-1}
        \begin{aligned}
            a_{k+1}
                & = (a_1 - \frac{q}{1-p})p^k + \frac{q}{1-p} \\
                & = (a_1 - 10 \varepsilon)(\frac{9}{10})^k + 10 \varepsilon \\
                & = (\frac{9}{10})^k a_1 - (\frac{9}{10})^k 10 \varepsilon + 10 \varepsilon \\
                & \leq (\frac{9}{10})^k (\frac{9}{10} a_0 + \varepsilon) - (\frac{9}{10})^k 10 \varepsilon + 10 \varepsilon \\
                & = (\frac{9}{10})^k \frac{9}{10} a_0 + (\frac{9}{10})^k \varepsilon - (\frac{9}{10})^k 10 \varepsilon + 10 \varepsilon \\
                & = (\frac{9}{10})^{k+1} a_0 + 10 \varepsilon \left(\frac{1}{10}(\frac{9}{10})^k - (\frac{9}{10})^k +1\right) \\
                & = (\frac{9}{10})^{k+1} a_0 + 10 \varepsilon \left((\frac{9}{10})^k (\frac{1}{10} - 1) +1\right) \\
                & = (\frac{9}{10})^{k+1} a_0 + 10 \varepsilon \left(- (\frac{9}{10})^k \frac{9}{10} +1\right) \\
                & = (\frac{9}{10})^{k+1} a_0 + 10 \varepsilon \left(1 - (\frac{9}{10})^{k+1}\right) \\
        \end{aligned}
    \end{equation}
\end{proofEnd}
By reducing \autoref{prop:f-convergence-mapping} to our case where is $f$
defined in \autoref{eq:definition-f-ipm} we have $p = \frac{9}{10}$, hence, $f$
is a contraction mapping. \autoref{thm:banach-fixed-point} states that $f$ has
an unique fixed point $\bar a$. Since
\begin{equation*}
    \lim_{k \rightarrow \infty} (\frac{9}{10})^{k} = 0 \, ,
\end{equation*}
The fixed point of $f$ is
\begin{equation*}
    \begin{aligned}
        \bar a
            & = \lim_{k \rightarrow \infty} a_{k} \\
            & = \lim_{k \rightarrow \infty} \left( (\frac{9}{10})^{k} a_0 + 10 \varepsilon \left(1 - (\frac{9}{10})^{k}\right) \right) \\
            & = 10\varepsilon
    \end{aligned}
\end{equation*}

It is important to stress that $y^*$ such that $\tilde \nabla \Phi(y^*) = 0$ is
never computed. Typically there is not a finite rank $r$ such that $a_r = \bar
a$. However, $a_r$ can be arbitrarily close to $\bar a$ for a finite $r$. We
choose $r$ such that, for instance, $a_r \leq \frac{11}{10} \bar a$, which
implies that for all $k > r$,
\begin{equation*}
    \|y_k - x^*\| \leq 11\varepsilon \, .
\end{equation*}
Thus, for all such $k$, Newton's steps will keep on oscillating inside the
$11\varepsilon$-ball around $x^*$. As such, as soon as we obtain a $y_k$ that is
within the $11\varepsilon$-ball around $x^*$, it is wise to stop the procedure
since doing one more step will produce a $y_{k+1}$ for which we have no way to
say whether it will be closer to $x^*$ than $y_k$ was. Using
\autoref{eq:valueof-ak-1}, we can easily show that
\begin{equation*}
    r \leq \log_p(\varepsilon^{-1}) \, ,
\end{equation*}
with $p = \frac{9}{10}$. Consequently, the stop-condition of \autoref{alg:ipm}
becomes $\|y_k - x^*\| \leq 11 \varepsilon$, and we can expect that to happen
after $\log_p(\varepsilon^{-1})$ Newton's steps.

\printbibliography

\appendix

\section{QRAM Model}\label{ap:qram}

To achieve the speed-up promised by the quantum algorithms presented hereby, we assume the existence of a quantum device able to run quantum subroutines on at most $O(\log N)$ qubits, where $N$ is the size of the problem or the input. \\

Besides, we assume an access to a Quantum Random Access Memory (QRAM) which is, as its classical analog, composed of an \emph{input} register, a \emph{memory} array and an \emph{output} register. The main variations are that the input and output registers are composed of qubits rather than bits. Thus, the quantum computer can address memory in superposition meaning that a superposition of inputs returns a superposition of outputs, so that one can design the following quantum unitary
\begin{equation*}
    \sum_j \lambda_j \ket{j}_{in} \ket{0}_{out}  \ \xrightarrow{QRAM \ access} \ \sum_j \lambda_j  \ket{j}_{in} \ket{v_j}_{out}  \ \ ,
\end{equation*}
where $in$, $out$ represent respectively the \emph{input} and the \emph{output} registers and $v_j$ the value contained in the $j-th$ register. Hence, a reading operation corresponds to a quantum query to the classical bits stored in the memory array, whereas the operation of writing a bit in the QRAM stays classical.\\

Within this computational model, the complexity of an algorithm can have several definitions.  One can consider either the \emph{time complexity}, which counts the number of elementary gates (classical and quantum), of quantum queries to the input and of QRAM operations, or the \emph{query complexity} which only counts the number of quantum queries to the input. As an example of actual QRAM, a quantum optical implementation is presented in \cite{QRAM}. 


\section{$k$-independent hash functions}\label{ap:def-k-independent}

\begin{definition}[$k$-independent hashing]\label{def:k-independent-hashing}
Let $\mathcal U$ be the set of keys. A family $\mathcal H = \big\{ h : \mathcal
U \rightarrow [m]\big\}$ is said to be $k$-independent if for all keys $x_1,
\cdots, x_k$ in $\mathcal U$ pairwise distinct and for all values $v_1, \cdots,
v_k$ in $[m]$,
\begin{equation*}
    \big| \{ h \in \mathcal H \; ;\; h(x_1)=v_1, \cdots,  h(x_k)=v_k \} \big| =
    \frac{|\mathcal H |}{m^k} \text{ ,}
\end{equation*}
in other words, by providing $\mathcal H$ with the uniform probability, for any
$h\in \mathcal H$
\begin{equation*}
    \mathbbm{P}\big(h(x_1)=v_1, \cdots,  h(x_k)=v_k \big) = \frac{1}{m^k} \text{ .}
\end{equation*}
\end{definition}






\end{document}

%%% Local Variables:
%%% mode: latex
%%% TeX-master: t
%%% End: