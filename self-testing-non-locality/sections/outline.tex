\setcounter{section}{0}

\setlength{\parindent}{0pt}

\section{Introduction}

[classical physics: locality + realism] + [qm: paradox (EPR) => qm incomplete]

[Bell 1964: qm => nonlocality]

[characterizing quantum vs non-quantum: violation of bell inequalities]

[applications for self-testing]

\section{Theoretical Framework}

[Alice Bob, Hilbert space, shared state => Bell Scenario] + [misc: qm + qi]

[behaviour / correlation]

[geometric view of correlations]

\subsection{Characterizing (non)local correlations}

[local correlations and LHV]

[no-signaling]

[quantum]

[convex polytope => LP can be used for optimization problems in it]

[chsh example + give correlations expression] =

\subsection{Self-testing of quantum states}

[minimum (necessary) self-testing defs]

...

[mayers yao self-test + proof + correlations expression]

\section{Non-locality detection}

[given prob dist (behaviour), say whether quantum]

\subsection{Linear Programming Formulation}

[introduce everything and explain]

[how to implement]

[how to interpret]

\subsection{CHSH correlations}

[specific formulation] + [results] + [interpretation]

\subsection{Mayer-Yao’s correlation}

[specific formulation] + [results] + [interpretation]


\section{Recovering Bell Inequalities}

[useful for self-test -> we can recover the state (or an isometry of that state) that maximally violates bell inequality -> test whether setup quantum]

\subsection{Formulation of the Dual}

[introduce dual, its interest and explain all variables]

[how to implement]

[how to interpret]

[other dual possibility]

\subsection{CHSH}

[specific formulation] + [results] + [interpretation]

\subsection{Mayer-Yao}

[specific formulation] + [results] + [interpretation]

\section{Restricted inputs for Mayer-Yao’s self-test}

[introduction: why is it useful]

[formulation]

[results + analysis]

\section{Robustness against white noise}
[with how much noise in a channel a state stays quantum]

[difference between CHSH and MY]

\section{Non-linear programming}
\subsection{Finding the optimal probability distribution}
[non linear program] + [procedure to find the P that maximally violates chsh]

\subsection{Making a behavior local}
[given a behavior, try to make it local while fixing k probabilities,
letting the other as variables]


\section{Discussion} 

\appendix
\section{mathematical optimization}

[linear programming => objective, constraints, solver we used, primal-dual, 
theorem strong-duality ]

[short sub chapter about non linear (quadratic) programming]

\section{Proofs we relied on}
[proof mayers yao we did at the beginning]