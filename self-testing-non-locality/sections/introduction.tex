\section{Introduction}\label{introduction}

Before the advent of quantum mechanics in the beginning of the \nth{20} century, physicists held three fundamental beliefs about the nature of the universe: 
\emph{determinism} (or \emph{causal determinism}), i.e.\ 
events in a given paradigm are bound by causality in such a way that any state is completely determined by prior states; 
\emph{reality}, i.e.\ the universe exists independently of observation; 
and \emph{locality} (or \emph{local causality}), i.e.\ what happens in a given space-time region should not influence what happens in another, space-like separated region. Quantum mechanics challenged each one of these notions \cite{nielsen_quantum_2010}.

Non-locality in particular posed the most problems: it was the corner-stone of Einstein, Podolsky and Rosen's argument in 1935 for the incompleteness of quantum mechanics \cite{epr}. 
Since it seemed to be possible for two spatially-separated parties sharing an entangled state to influence each other's measurements, resulting in \emph{faster-than-light communications}, they argued that such a behaviour was paradoxical (this became known as the \emph{EPR paradox}), and postulated that quantum mechanics should be compatible with a \emph{hidden local variable description}, in which non-local effects would not be allowed.

For more than 30 years, \emph{local-realist} description of quantum 
theory were not fully discarded. 
One had to wait until 1964, when \citeauthor{bell_einstein_1964}
%noticed that certain quantum correlations could not be accounted for by any theory which attributes only locally defined states to its basic physical objects\cite{bell_einstein_1964}.
proposed an experiment which would definitively decide whether or not certain physical effects of quantum entanglement could be reproduced
by local hidden variables: the \emph{Bell test} \cite{bell_einstein_1964}. He proposed a condition, known as \emph{Bell’s inequality}, that any physical experiment
has to satisfy if nature could be faithfully described by a classical local hidden variable theory. Subsequently, several experiments
% \hft{lesquelles ?} 
\cite{aspect_experiment} have been designed that have violated Bell’s inequality, proving 
that no hidden variable theory can describe certain phenomenons predicted by quantum mechanics.

Since the 1960s, the field of \emph{Bell nonlocality} has grown quite considerably \cite{brunner}, especially with the establishment of \emph{quantum information science}, where nonlocality plays a central role \cite{nielsen_quantum_2010}. It is at the heart of \emph{device-independent} (\emph{d.i.}) protocols such as \emph{self-testing} \cite{2020-self-testing-a-review}, \emph{d.i.\ quantum key distribution} \cite{di-qkd}, \emph{d.i.\ randomness generation} \cite{di-random}, and \emph{delegated blind quantum computing} \cite{blind-aaronson}.

%According to them, the formulation of quantum mechanics seemed to imply faster than light communication. 

% and it wasn't until the work of Bell in 1964 that it started being widely accepted as a fundamental consequence of the theory.

% Quantum mechanics challenged each one of these notions: in 1927, W. Heisenberg introduced the uncertainty principle which gives a fundamental limit to the accuracy certain pairs of physical quantities can be determined to from initial conditions; the Copenhagen interpretation 

% reality was put into question as well, since a particle can exist in superposition and only by observing it can you determine 

% Quantum mechanics challenged each one of these notions: [Heisenberg uncertainty principle => a precise knowledge at the quantum level is impossible] [wave-like properties of matter: things can exist in a supperposition of different states. (copenhagen interpretation of quantum mechanics)] [non-locality: space-like seperated states can exhibit correlations]

% Non-locality especially posed several problems: it was the corner-stone of Einstein, Podolsky and Rosen's argument in 1935 for the incompleteness of quantum mechanics:.


% locality = what happens in a given space-time region should not influence what 
% happens in another, space-like separated regiont
% determinism = 
% reality = the universe exists independent of observation

% [qm: paradox (EPR) => qm incomplete]
% [observations predited by quantum mechanics forced physicists to abandon these notions => neither locality neither realism hold]


% Quantum mechanics challenged each one of these notions: non-locality especially posed
% several problems, and was the corner-stone of Einstein's argument in 1935 for the 
% incompleteness of quantum mechanics. They argued that quantum theory should allow a 
% description with local hidden variables.

% [Bell 1964: qm => nonlocality]
% In 1964, Bell showed that quantum theory is incompatible with local hidden variables. 
% Since then, nonlocality has been studied extensively - notably in quantum information. 

% [self-testing]
% Several applications exist: notably self-testing and device-independent scenarios use
% expressions of nonlocality extensively.

% \hpa{put paragraph on applications: device-independent quantum key, randomness generation, self-test...}

\Autoref{seq:theoretical-framework} sets the stage for the rest of the report, establishing the relevant theoretical background and framework. Subsequently, deeper attention is given to the two main bodies of work: nonlocality detection (\Autoref{sec:nonloc-detection}) and the retrieval of Bell inequalities (\Autoref{seq:ineq-dual}). In \Autoref{sec:noise}, the limits of locality detection in noisy systems are studied, and lastly, in \Autoref{sec:non-linear}, a nonlinear approach to study of nonlocality is described.