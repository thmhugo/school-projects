\section{Non-linear programming}\label{sec:non-linear}

Considering \sref{Linear Program}{eq:primal}, one can transform the vector
$\mathcal P$ into a vector of variables. This converts the initial program into 
a nonlinear program, more specifically a quadratic program, and provides as a result a behavior $\mathcal P$. This kind of transformation adds new 
difficulties for the resolution, such as non-convex constraints, but all of these 
are left over to the solver.

\subsection{Finding the optimal probability distribution}

Adding the constraints for $\mathcal P$
to be a probability vector yields the following nonlinear program
\begin{equation} \label{eq:primal-nl}
	\begin{aligned}
			  & \max \; \beta_{CHSH}\\
		s.t.  &
		\begin{cases}
			(1-q) \mathcal P + q \mathbbm{1}
                & = \sum_\lambda \mu_\lambda \mathbf{d}_\lambda \\
			\sum_\lambda \mu_\lambda & = 1 \\
            \sum_{a,b} p(a,b|x,y) & = 1, \; \forall (x,y) \in \{0,1\}^2 \\
			q & \leq 1 \\
		    \mu_\lambda \geq 0 \; \forall \lambda, \; q \geq 0
		\end{cases}
	\end{aligned}
\end{equation}
that one could solve to find the maximal value for $\beta_{CHSH}$, whose value is
the given by the left-hand side of \Autoref{eq:chsh}. However, 
it must be remembered that the maximal value one can obtain doing this is
$\beta_{CHSH} = 4$, where the resulting probability distribution describes a 
behavior belonging to $\NS \setminus \Q$.

Finding the boundaries between the sets $\Q$ and $\NS$ i.e., inequalities
violated by behaviors in $\NS \setminus \Q $, is an open problem when
considering arbitrary values of $\Delta$ and $m$. For bipartite scenarios with
binary outcomes ($\Delta = 2$, $m = 2$), \Autoref{eq:bound-ns-q}
\begin{equation} \label{eq:bound-ns-q}
    | \asin \braket{A_0 B_0}
    + \asin \braket{A_0 B_1}
    + \asin \braket{A_1 B_0}
    - \asin \braket{A_1 B_1} | \leq \pi,
\end{equation}
introduced by \citeauthor{Masanes2003NecessaryAS} in \cite{Masanes2003NecessaryAS}, 
provides an explicit description of behaviours
belonging to $\Q$. Moreover, the two sides are equal whenever the behavior
maximally violates the CHSH \sref{Inequality}{eq:chsh}.

While simple to understand, this constraint cannot be enforced by Gurobi, that
cannot handle constraints described by trigonometric functions. Nonetheless, it
will only be used to check whether a behavior stays in $\Q$. 

Taking into consideration both \sref{Quadratic Program}{eq:primal-nl} and
\Autoref{eq:bound-ns-q} yields the following procedure to find the
maximal value of $\beta_{CHSH}$

\begin{algorithm}[H]
\caption{Numerical upper bound for CHSH inequality violation}
\begin{algorithmic}[1]
    \Require $\delta$  is the precision, roughly corresponds to the number of number of significant digits
    \Require $\Pi = 2$ is the starting upper bound
    \While{$|\sum_{x,y} (-1)^{xy} \asin \braket{A_x B_y}| \leq \pi$}
        \State $\Pi \leftarrow \Pi + \delta$
        \State \textit{solve \sref{linear program}{eq:primal-nl} with constraint $\beta_{CHSH} \leq \Pi$}
    \EndWhile
    \State \Return $\mathcal P$\textit{: the optimal probability distribution}
\end{algorithmic}
\label{alg:beta-optimal-chsh}
\end{algorithm}

The numerical result is, indeed, arbitrary close to $2 \sqrt 2$ depending on
$\delta$. One should note that not such equation as
\Autoref{eq:bound-ns-q} is known for the case $\Delta = 2$ and $m=3$, but it is
possible to use the generalized CHSH inequality as depicted by \citeauthor{wehner_tsirelson_2006} in
\cite{wehner_tsirelson_2006}. We could have use this
for the case $\Delta = m = 2$ even though it will yield
the same result, but we thought it was more interesting to try at least once another 
formulation. Considering the case $\Delta = 2$ and $m=3$, we indeed obtained results 
arbitrary close to $3 \sqrt 3$, and the associated probability distribution.

\subsection{Making a behavior local}

Looking for the optimal probability distribution with a non-linear approach led
us to a side work, that is, from knowing how many elements of the behavior one 
can infer that the behavior is local. Taking into consideration that a behavior with
$\Delta = m = 2$ is local if and only if $\beta_{CHSH} \leq 2$, it suffices to
set $k$ probabilities of, for instance, the behavior that maximally violates the
CHSH inequality, and solving \sref{Nonlinear Program}{eq:primal-nl}, where the
maximization becomes a minimization. By doing this, if the optimal value is a
$\beta^*_{CHSH} \leq 2$, then it is possible to find a full behavior that is local,
and its elements are given in the same time by the solver. Otherwise, one
can be sure that it is not possible to make the behavior local.

The results of the implementation with Gurobi yielded on one hand that with $k =
4$, where for each $(x,y)$, a single $p(a,b|x,y)$ is fixed, it is not anymore
possible to find values for the other probabilities such that the behavior is
local, in addition, it only results on the behavior that gives $\beta_{CHSH} = 2
\sqrt 2$, since it removes all degrees of freedom for the other probabilities
in the light of the normalization constraints (\Autoref{eq:norm_cons_1}) and 
(\Autoref{eq:norm_cons_2}).

Moreover, fixing only one expectation $\braket{A_x B_y}$ gives an
optimal of 0 with $\beta_{CHSH} = 0$, while fixing an arbitrary number of 
probabilities among up to 3 expectations yields a local behavior, with an optimal
0, yet $\beta_{CHSH} \approx 1.1213$.

Besides, thanks to the implementation of \sref{Nonlinear Program}{eq:primal-nl}, we were able to
generate a behavior $\mathcal P_{\Delta}$ whose CHSH value is $\beta_{CHSH} = 2
+ \Delta$, hence that violates the inequality with an arbitrary small precision 
$\Delta$. It yielded for the both aforementioned cases exactly the same results.

For the case of the Mayers-Yao's correlations, the same result is obtained when
fixing one probability per expectation i.e., the resulting probability distribution
stays unchanged. One the other hand, fixing 3 arbitrary expectation also yields 
the same optimal objective as given in \Autoref{eq:opt-primal-my}, enforcing the 
behavior to be in $\Q$, while fixing only 2 arbitrary expectations causes the
behavior to be local.