\documentclass[10pt, letterpaper, twoside, twocolumn]{article}
\usepackage[T1]{fontenc}
\usepackage[utf8]{inputenc}
\usepackage{ geometry }
\geometry{margin=1.5cm, top=2cm, bottom=2cm}
\usepackage{ amssymb }
\usepackage{ bbm }
\usepackage{ braket }
\usepackage{ mathrsfs }
\usepackage{ geometry }
\usepackage{ amsmath }
%\usepackage[alphabetic]{ amsrefs }
\usepackage{ hyperref }
\usepackage{ enumitem }
\usepackage{ graphicx }
\setlength{\columnsep}{0.8cm}
\usepackage{ blindtext }
\usepackage{ caption }
\usepackage{ bookmark }
\usepackage{ float }
\usepackage{ pgf }
\usepackage{ tikz }
\usepackage{pgfplots}
\usetikzlibrary{graphs,graphs.standard,quotes,arrows.meta}
\usepackage{ authblk }
\usepackage{ sectsty } % section/subection font size
\usepackage{ algorithm }
\usepackage{ algpseudocode }
\usepackage{ amssymb }
\usepackage{ bbm }
\usepackage{ braket }
\usepackage{ mathrsfs }
\usepackage{comment}
\usepackage{amsthm}
\usepackage{xcolor}
\usepackage{ geometry }
\usepackage{ amsmath }
%\usepackage[alphabetic]{amsrefs}
\usepackage{ hyperref }
\usepackage{ bbm }
\usepackage{ diagbox }
\usepackage{ multirow }
\usepackage[super]{nth}
\usepackage{authblk}
\usepackage{cleveref}

\usepackage[style=alphabetic,natbib=true,backend=bibtex]{biblatex}
\usepackage{refcount}

\AtEveryBibitem{
    \clearfield{urlyear}
    \clearfield{urlmonth}
    \clearfield{url}
    % \clearfield{doi}
    \clearfield{issn}
}

\addbibresource{./misc/bibliography.bib} %Imports bibliography file

\usetikzlibrary{shapes.geometric}
\usetikzlibrary{decorations}
\usetikzlibrary{arrows.meta}

\usepackage{ fancyhdr }
\pagestyle{fancy}
\fancyhf{}
\fancyhead[LOE]{\leftmark}
\fancyfoot[COE]{\thepage}

\fancypagestyle{firststyle}
{
   \fancyhf{}
   \fancyfoot[C]{\thepage}
   \renewcommand{\headrulewidth}{0pt} % removes horizontal header line
   \newgeometry{margin=1.5cm, bottom=2cm}
}

\theoremstyle{plain}
\newtheorem{theorem}{Theorem}[section]
\newtheorem{corollary}[theorem]{Corollary}
\newtheorem{lemma}[theorem]{Lemma}
\newtheorem{proposition}[theorem]{Proposition}

\theoremstyle{definition}
\newtheorem{definition}[theorem]{Definition}
\newtheorem{example}[theorem]{Example}
\newtheorem{procedure}[theorem]{Procedure}
\newtheorem{property}[theorem]{Property}
\newtheorem{notation}[theorem]{Notation}
\newtheorem{convention}[theorem]{Convention}
\newtheorem{fact}{Fact}
\newtheorem{interpretation}[theorem]{Interpretation}
\newtheorem{remark}[theorem]{Remark}
% \newtheorem{question}[theorem]{Question}
\newtheorem{note}[theorem]{Note}

\newtheorem{question}{Question}
\theoremstyle{plain}
\newtheorem{assertion}{Assertion}[question]

\newcommand{\sref}[2]{\hyperref[#2]{#1 \ref{#2}}}



\renewcommand{\epsilon}{\varepsilon}
\newcommand{\x}{\chi}

% Figures in muticols.
% Captions with \captionof{figure}{}
\newenvironment{Figure}
  {\par\medskip\noindent\minipage{\linewidth}}
  {\endminipage\par\medskip}

% Flush left the title and authors.
\makeatletter
\renewcommand{\maketitle}{\bgroup\setlength{\parindent}{0pt}
\begin{flushleft}
  \Large{\textbf{\@title}}

  \normalsize\@author

\hspace*{3.1cm}  \@date

\end{flushleft}\egroup
}
\makeatother

\renewcommand\Authands{ and }

\sectionfont{\fontsize{12}{12}\selectfont}
\subsectionfont{\fontsize{11}{11}\selectfont}

\renewcommand{\L}{\mathcal{L}}
\newcommand{\NS}{\mathcal{NS}}
\newcommand{\Q}{\mathcal{Q}}
\newcommand{\asin}{\text{asin}}

\newcommand{\hpa}[1]{\textcolor{teal}{#1}}
\newcommand{\ft}[1]{\textcolor{blue}{#1}}
\newcommand{\hft}[1]{\textcolor{red}{#1}}

\renewenvironment{abstract}
{\small
  \begin{center}
    \bfseries Abstract\vspace{-.5em}\vspace{0pt}
  \end{center}
  \list{}{
    \setlength{\leftmargin}{0mm}
    \setlength{\rightmargin}{\leftmargin}
  }
\item\relax}
{\endlist}


\makeatletter
\renewcommand*{\@seccntformat}[1]{\csname the#1\endcsname\hspace{0.4cm}}
\makeatother



\usepackage{catoptions}
\makeatletter
\def\figureautorefname{figure}
\def\tableautorefname{table}
\def\Autoref#1{%
  \begingroup
  \edef\reserved@a{\cpttrimspaces{#1}}%
  \ifcsndefTF{r@#1}{%
    \xaftercsname{\expandafter\testreftype\@fourthoffive}
      {r@\reserved@a}.\\{#1}%
  }{%
    \ref{#1}%
  }%
  \endgroup
}
\def\testreftype#1.#2\\#3{%
  \ifcsndefTF{#1autorefname}{%
    \def\reserved@a##1##2\@nil{%
      \uppercase{\def\ref@name{##1}}%
      \csn@edef{#1autorefname}{\ref@name##2}%
      \autoref{#3}%
    }%
    \reserved@a#1\@nil
  }{%
    \autoref{#3}%
  }%
}
\makeatother

\makeatletter
\newcommand\autorefs[1]{\@first@ref#1,@}
\def\@throw@dot#1.#2@{#1}% discard everything after the dot
\def\@set@refname#1{%    % set \@refname to autoefname+s using \getrefbykeydefault
    \edef\@tmp{\getrefbykeydefault{#1}{anchor}{}}%
    \xdef\@tmp{\expandafter\@throw@dot\@tmp.@}%
    \ltx@IfUndefined{\@tmp autorefnameplural}%
         {\def\@refname{\@nameuse{\@tmp autorefname}s}}%
         {\def\@refname{\@nameuse{\@tmp autorefnameplural}}}%
}
\def\@first@ref#1,#2{%
  \ifx#2@\autoref{#1}\let\@nextref\@gobble% only one ref, revert to normal \autoref
  \else%
    \@set@refname{#1}%  set \@refname to autoref name
    \@refname~\ref{#1}% add autoefname and first reference
    \let\@nextref\@next@ref% push processing to \@next@ref
  \fi%
  \@nextref#2%
}
\def\@next@ref#1,#2{%
   \ifx#2@ and~\ref{#1}\let\@nextref\@gobble% at end: print and+\ref and stop
   \else, \ref{#1}% print  ,+\ref and continue
   \fi%
   \@nextref#2%
}

\makeatother

\usepackage[small]{titlesec}

\DeclareFieldFormat*{citetitle}{\textit{#1}}
\DeclareCiteCommand{\citejournal}
  {\usebibmacro{prenote}}
  {\usebibmacro{citeindex}%
    \usebibmacro{journal}}
  {\multicitedelim}
  {\usebibmacro{postnote}}


\newcommand{\definitionautorefname}{Definition}
\newcommand{\propositionautorefname}{Proposition}

\usetikzlibrary{calc}

\tikzset{%
  add/.style args={#1 and #2}{to path={%
 ($(\tikztostart)!-#1!(\tikztotarget)$)--($(\tikztotarget)!-#2!(\tikztostart)$)%
  \tikztonodes}}
}