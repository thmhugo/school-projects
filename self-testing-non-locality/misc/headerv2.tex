\documentclass[10pt, letterpaper, twocolumn]{article}
\usepackage[T1]{fontenc}
\usepackage[utf8]{inputenc}
\usepackage{ geometry }
\geometry{margin=1.5cm, top=2cm, bottom=2cm}
\usepackage{ amssymb }
\usepackage{ bbm }
\usepackage{ braket }
\usepackage{ mathrsfs }
\usepackage{ geometry }
\usepackage{ amsmath }
%\usepackage[alphabetic]{ amsrefs }
\usepackage{ hyperref }
\usepackage{ enumitem }
\usepackage{ graphicx }
\setlength{\columnsep}{0.8cm}
\usepackage{ blindtext }
\usepackage{ caption }
\usepackage{ bookmark }
\usepackage{ float }
\usepackage{ stfloats }
\usepackage{ pgf }
\usepackage{ tikz }
\usepackage{ pgfplots }
\usetikzlibrary{graphs,graphs.standard,quotes,arrows.meta}
\usepackage{ authblk }
\usepackage{ sectsty } % section/subection font size
\usepackage{ algorithm }
\usepackage[noend]{ algpseudocode }
\usepackage{ amssymb }
\usepackage{ bbm }
\usepackage{ braket }
\usepackage{ mathrsfs }
\usepackage{comment}
\usepackage{amsthm}
\usepackage{xcolor}
\usepackage{ geometry }
\usepackage{ amsmath }
%\usepackage[alphabetic]{amsrefs}
\usepackage{ hyperref }
\usepackage{ bbm }
\usepackage{ diagbox }
\usepackage{ multirow }
\usepackage[super]{nth}
\usepackage{authblk}
\usepackage{cleveref}
\usepackage{proof-at-the-end}
\usepackage[english]{babel}

\usepackage[style=alphabetic,natbib=true,backend=bibtex]{biblatex}
\usepackage{refcount}

\AtEveryBibitem{
    \clearfield{urlyear}
    \clearfield{urlmonth}
    \clearfield{url}
    % \clearfield{doi}
    \clearfield{note}
    \clearfield{issn}
}

\addbibresource{misc/bibliography.bib} %Imports bibliography file

\usetikzlibrary{shapes.geometric}
\usetikzlibrary{decorations}
\usetikzlibrary{arrows.meta}

\usepackage{ fancyhdr }
\pagestyle{fancy}
\fancyhf{}
\fancyhead[LOE]{\leftmark}
\fancyfoot[COE]{\thepage}

\fancypagestyle{firststyle}
{
   \fancyhf{}
   \fancyfoot[C]{\thepage}
   \renewcommand{\headrulewidth}{0pt} % removes horizontal header line
   \newgeometry{margin=1.5cm, bottom=2cm}
}

\theoremstyle{plain}
\newtheorem{theorem}{Theorem}[section]
\newtheorem{corollary}[theorem]{Corollary}
\newtheorem{lemma}[theorem]{Lemma}
\newtheorem{proposition}[theorem]{Proposition}

\theoremstyle{definition}
\newtheorem{definition}[theorem]{Definition}
\newtheorem{example}[theorem]{Example}
\newtheorem{procedure}[theorem]{Procedure}
\newtheorem{property}[theorem]{Property}
\newtheorem{notation}[theorem]{Notation}
\newtheorem{convention}[theorem]{Convention}
\newtheorem{fact}{Fact}
\newtheorem{interpretation}[theorem]{Interpretation}
\newtheorem{remark}[theorem]{Remark}
% \newtheorem{question}[theorem]{Question}
\newtheorem{note}[theorem]{Note}
\newtheorem{claim}{Claim}
\newtheorem{question}{Question}
\theoremstyle{plain}
\newtheorem{assertion}{Assertion}[question]

\newcommand{\sref}[2]{\hyperref[#2]{#1 \ref{#2}}}



\renewcommand{\epsilon}{\varepsilon}
\newcommand{\x}{\chi}

% Figures in muticols.
% Captions with \captionof{figure}{}
\newenvironment{Figure}
  {\par\medskip\noindent\minipage{\linewidth}}
  {\endminipage\par\medskip}

% Flush left the title and authors.
\makeatletter
\renewcommand{\maketitle}{\bgroup\setlength{\parindent}{0pt}
\begin{flushleft}
  \begin{center}\Large{\centering\textbf{\@title}}\end{center}
  \normalsize\@author
  \vspace{.5cm}
  \begin{center}\@date\end{center}

\end{flushleft}\egroup
}
\makeatother

\renewcommand\Authands{ and }

\sectionfont{\fontsize{12}{12}\selectfont}
\subsectionfont{\fontsize{11}{11}\selectfont}


\newcommand{\hpa}[1]{\textcolor{teal}{#1}}
\newcommand{\ft}[1]{\textcolor{blue}{#1}}
\newcommand{\hft}[1]{\textcolor{purple}{#1}}

%%%%%%%%%%%
% Autoref %
%%%%%%%%%%%

% Rename section, subsection, subsubsection into Section
\addto\extrasenglish{%
  \def\sectionautorefname{Section}
  \def\subsectionautorefname{Section}
  \def\subsubsectionautorefname{Section}
  \def\algorithmautorefname{Algorithm}
}

\renewenvironment{abstract}
{\small
  \begin{center}
    \bfseries Abstract\vspace{-.5em}\vspace{0pt}
  \end{center}
  \list{}{
    \setlength{\leftmargin}{0mm}
    \setlength{\rightmargin}{\leftmargin}
  }
\item\relax}
{\endlist}


\usepackage[small]{titlesec}

\DeclareFieldFormat*{citetitle}{\textit{#1}}
\DeclareCiteCommand{\citejournal}
  {\usebibmacro{prenote}}
  {\usebibmacro{citeindex}%
    \usebibmacro{journal}}
  {\multicitedelim}
  {\usebibmacro{postnote}}


\newcommand{\<}{\leftarrow}
\renewcommand{\>}{\rightarrow}

\newcommand{\bs}{\boldsymbol}
\newcommand{\A}{\boldsymbol A}
\newcommand{\B}{\boldsymbol B}

\newcommand{\esp}[1]{\mathbb E \left[ #1 \right]} % Command for E[..]

% \usepackage{titlesec}

% \renewcommand{\thesection}{\Roman{section}}
% \renewcommand{\thesubsection}{\thesection.\Roman{subsection}}

% % \titleformat*{\section}{\large\bfseries}
% \titleformat{\section}
%   {\large\scshape\filcenter}{\thesection}{1em}{}
% \titleformat{\subsection}
%   {\normalsize\scshape\filcenter}{\thesubsection}{1em}{.}
% \titleformat{\subsubsection}
%   {\normalsize\scshape\filcenter}{\thesubsubsection}{1em}{}
\usetikzlibrary{calc}


\tikzset{%
  add/.style args={#1 and #2}{to path={%
 ($(\tikztostart)!-#1!(\tikztotarget)$)--($(\tikztotarget)!-#2!(\tikztostart)$)%
  \tikztonodes}}
}



\renewcommand{\L}{\mathcal{L}}
\newcommand{\NS}{\mathcal{NS}}
\newcommand{\Q}{\mathcal{Q}}
\newcommand{\asin}{\text{asin}}